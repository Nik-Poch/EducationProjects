\documentclass[a4paper,14pt]{article}

% В этом документе преамбула

%%% Работа с русским языком
\usepackage{cmap}					% поиск в PDF
\usepackage{mathtext} 				% русские буквы в формулах
\usepackage[T2A]{fontenc}			% кодировка
\usepackage[utf8]{inputenc}			% кодировка исходного текста
\usepackage[english,russian]{babel}	% локализация и переносы
\usepackage{indentfirst}			% чтобы первый абзац в разделе отбивался красной строкой
\frenchspacing						% тонкая настройка пробелов

%%% Приведение начертания букв и знаков к русской типографской традиции
\renewcommand{\epsilon}{\ensuremath{\varepsilon}}
\renewcommand{\phi}{\ensuremath{\varphi}}			% буквы "эпсилон"
\renewcommand{\kappa}{\ensuremath{\varkappa}}		% буквы "каппа"
\renewcommand{\le}{\ensuremath{\leqslant}}			% знак меньше или равно
\renewcommand{\leq}{\ensuremath{\leqslant}}			% знак меньше или равно
\renewcommand{\ge}{\ensuremath{\geqslant}}			% знак больше или равно
\renewcommand{\geq}{\ensuremath{\geqslant}}			% знак больше или равно
\renewcommand{\emptyset}{\varnothing}				% знак пустого множества

%%% Дополнительная работа с математикой
\usepackage{amsmath,amsfonts,amssymb,amsthm,mathtools} % AMS
\usepackage{icomma} % "Умная" запятая: $0,2$ --- число, $0, 2$ --- перечисление

%% Номера формул
\mathtoolsset{showonlyrefs=true} % Показывать номера только у тех формул, на которые есть \eqref{} в тексте.

%% Свои команды

% операции, не определённые (или имеющие иные обохначения) в мат. пакетах
\DeclareMathOperator{\sgn}{\mathop{sgn}}				% ф-ия sgn
\renewcommand{\tg}{\mathop{\mathrm{tg}}\nolimits}		% обозначение тангенса

%% Перенос знаков в формулах (по Львовскому)
\newcommand*{\hm}[1]{#1\nobreak\discretionary{}
{\hbox{$\mathsurround=0pt #1$}}{}}

%%% Работа с картинками
\usepackage{graphicx}  % Для вставки рисунков
\graphicspath{{images/}{images2/}}  % папки с картинками
\setlength\fboxsep{3pt} % Отступ рамки \fbox{} от рисунка
\setlength\fboxrule{1pt} % Толщина линий рамки \fbox{}
\usepackage{wrapfig} % Обтекание рисунков текстом

%%% Работа с таблицами
\usepackage{array,tabularx,tabulary,booktabs} % Дополнительная работа с таблицами
\usepackage{longtable}  % Длинные таблицы
\usepackage{multirow} % Слияние строк в таблице

%%% Теоремы
\theoremstyle{plain} % Это стиль по умолчанию, его можно не переопределять.
\newtheorem{theorem}{Теорема}[section]
\newtheorem{lemma}{Лемма}[section]
\newtheorem{definition}[theorem]{Определение}
\newtheorem{property}{Свойство}
 
\theoremstyle{definition} % "Определение"
\newtheorem{corollary}{Следствие}[theorem]
\newtheorem{exmp}{Пример}[section]
 
\theoremstyle{remark} % "Примечание"
\newtheorem*{nonum}{Решение}
\newtheorem*{evidence}{Доказательство}
\newtheorem*{remark}{Примечание}

%%% Программирование
\usepackage{etoolbox} % логические операторы

%%% Страница
\usepackage{extsizes} % Возможность сделать 14-й шрифт
\usepackage{geometry} % Простой способ задавать поля
	\geometry{top=25mm}
	\geometry{bottom=35mm}
	\geometry{left=35mm}
	\geometry{right=20mm}

%\usepackage{fancyhdr} % Колонтитулы
% 	\pagestyle{fancy}
 	%\renewcommand{\headrulewidth}{0pt}  % Толщина линейки, отчеркивающей верхний колонтитул
% 	\lfoot{Нижний левый}
% 	\rfoot{Нижний правый}
% 	\rhead{Верхний правый}
% 	\chead{Верхний в центре}
% 	\lhead{Верхний левый}
%	\cfoot{Нижний в центре} % По умолчанию здесь номер страницы

\usepackage{setspace} % Интерлиньяж (межстрочные интервалы)
%\onehalfspacing % Интерлиньяж 1.5
%\doublespacing % Интерлиньяж 2
%\singlespacing % Интерлиньяж 1

\usepackage{lastpage} % Узнать, сколько всего страниц в документе.

\usepackage{soulutf8} % Модификаторы начертания

\usepackage{hyperref}
\usepackage[usenames,dvipsnames,svgnames,table,rgb]{xcolor}
\hypersetup{				% Гиперссылки
    unicode=true,           % русские буквы в раздела PDF
    pdftitle={Заголовок},   % Заголовок
    pdfauthor={Автор},      % Автор
    pdfsubject={Тема},      % Тема
    pdfcreator={Создатель}, % Создатель
    pdfproducer={Производитель}, % Производитель
    pdfkeywords={keyword1} {key2} {key3}, % Ключевые слова
    colorlinks=true,       	% false: ссылки в рамках; true: цветные ссылки
    linkcolor=MidnightBlue,          % внутренние ссылки
    citecolor=black,        % на библиографию
    filecolor=magenta,      % на файлы
    urlcolor=blue           % на URL
}

\usepackage{csquotes} % Еще инструменты для ссылок

%\usepackage[style=authoryear,maxcitenames=2,backend=biber,sorting=nty]{biblatex}

\usepackage{multicol} % Несколько колонок

%%% Работа с графикой
\usepackage{tikz}
\usetikzlibrary{calc}
\usepackage{tkz-euclide}
\usetikzlibrary{arrows}
\usepackage{pgfplots}
\usepackage{pgfplotstable}

%%% Настройка подписей к плавающим объектам
\usepackage{floatrow}	% размещение
\usepackage{caption}	% начертание
\captionsetup[figure]{labelfont=bf,textfont=it,font=footnotesize}	% нумерация и надпись курсивом
% для подфигур: заголовок подписи полужирный, текст заголовка обычный
% выравнивание является неровным (т.е. выровненным по левому краю)
% singlelinecheck = off означает, что настройка выравнивания используется, даже если заголовок имеет длину только одну строку.
% если singlelinecheck = on, то заголовок всегда центрируется, когда заголовок состоит только из одной строки.
\captionsetup[subfigure]{labelfont=bf,textfont=normalfont,singlelinecheck=off,justification=raggedright}

%%% Stuff для графиков и рисунков



%%% Ф-ии для АиГ'а
\DeclareMathOperator{\realpart}{\mathop{Re}}	% действительная часть
\DeclareMathOperator{\imgpart}{\mathop{Im}} 	% мнимая часть

\begin{document} % начало документа

\thispagestyle{empty}
\begin{center}
	\textbf{Санкт-Петербургский государственный электротехнический университет \\ <<,,\hspace{0.5pt}ЛЭТИ\hspace{0.5pt}`` имени В. И. Ульянова (Ленина)>>}
\end{center}
% выделение имени, тут отступ
\vspace{13ex}
\begin{flushright} % окружение выравнивания по правому краю
	\noindent % убираем отступ для красной строки
	\textit{Почаев Никита Алексеевич}
	\\
	\textit{студент ФКТИ \\(группа 8381)}
\end{flushright}
\begin{center}
	\vspace{13ex}
	\so{\textbf{КОНСПЕКТ}}
	\vspace{1ex}
	
	по линейной алгебре и геометрии
	
	
	по темам "Комплексные числа" и "Матрицы"
	
	
	за 1-ый семестр
	
	
	\vfill % вертикальный пробел, чтобы всё, что до него и до перехода на новую страницу оказалось внизу страницы
	{\small Санкт-Петербург, 2019}
\end{center}
\newpage
\section{Комплексные числа.}
\subsection{Алгебраическая и геометрическая формы комплексного числа, действия над комплексными числами в алгебраической форме}
Запись вида $z=a+bi, i^2=-1$ называется алгебраической или координатной формой комплексного числа $z$, где $a$ - его действительная часть ($\realpart z$), $b$ - мнимая часть ($\imgpart z$), а $i$ - мнимая единица.
\begin{figure}[!htbp]
	\begin{center}
		\begin{tikzpicture}[scale=0.8]
		\begin{scope}[thick,font=\scriptsize]
		
		\draw [->] (-4,0) -- (5,0) node [above left]  {$\realpart\{z\}$};
		\draw [->] (0,-4) -- (0,4) node [below right] {$\imgpart\{z\}$};
		
		\foreach \n in {-3,...,-1,1,2,...,3}{%
			\draw (\n,-3pt) -- (\n,3pt)   node [above] {$\n$};
			\draw (-3pt,\n) -- (3pt,\n)   node [right] {$\n i$};}
		
		\draw [thick, color=red] (0,0) -- (2,3);
		\draw [color=blue, fill=blue] (2,3) circle(0.05);
		\node [color=black] at (3,3) {$ 2+3i$};
		\end{scope}
		\end{tikzpicture}
	\end{center} \caption{изображение $2+3i$ на комплексной плоскости}
\end{figure}
\noindent Арифметические действия над комплексными числами:
\begin{enumerate}
	\item Сложение/вычитание:
	
	$z_1\pm\ z_2=\left(a+bi\right)\pm\left(c+di\right)=\left(a\pm c\right)+\left(b\pm d\right)i$
	\item Произведение:
	
	$z_1\cdot\ z_2=\left(a\ +\ bi\right)\ \cdot (c + di) = (ac - bd) + (ad + bc)i$
	\item Частное:
	
	$\dfrac{z_1}{z_2}=\dfrac{a+bi}{c+di}=\dfrac{\left(a+bi\right)\left(c-di\right)}{\left(c+di\right)\left(c-di\right)}=\dfrac{\left(ac+bd\right)+\left(bc-ad\right)i}{c^2+d^2}$
	\item Комплексно сопряжённое число:
	
	Для $z=a+bi \text{: } \bar{z}=a-bi$. При этом $z\cdot\bar{z}=a^2+b^2$.
	\item Степени мнимой единицы:
	
	\[
	i^n=\left\{
	\begin{aligned}
	i&, n=4k+1 \\
	-1&, n=4k+2 \\
	-i&, n=4k+3 \\
	1&, n=4k
	\end{aligned}
	\right.
	\]
\end{enumerate}
	
\subsection{Особенности геометрического представление комплексного числа на плоскости. Модуль и аргумент.}
Изоморфизм (совпадение форм): сложение комплексных чисел соответствует сложению векторов на плоскости.
Радиус вектор (или модуль) комплексного числа равен: $\left|z\right|=\sqrt{a^2+b^2}$
Модуль разности двух комплексных чисел есть расстояние между точками на комплексной плоскости, которым соответствуют этим числам.
\begin{figure}[!htbp]
	\begin{center}
		\begin{tikzpicture}
			\begin{scope}[thick,font=\scriptsize]
				% Оси:
				% Просто линии, нарисованные с опцией `->`, превращает их в стрелки:
				\draw [->] (-4,0) -- (5,0) node [above left]  {$\realpart z$};
				\draw [->] (0,-4) -- (0,4) node [below right] {$\imgpart z$};
				
				% рисуем параллелограмм
				\draw[dashed] (0, 0)--(3,2) node[circle,fill,inner sep=1pt]{};
				\draw[dashed] (3, 2)--(2,3) node[circle,fill,inner sep=1pt]{};
				\draw[dashed] (-1, 1)--(2,3) node[circle,fill,inner sep=1pt]{};
				\draw[dashed] (0,0)--(-1,1) node[circle,fill,inner sep=1pt]{};        
				
				% отмечаем параллелограмм
				\draw (-1, 1) node[circle,fill,inner sep=1pt,label=left:$-1+i$]{};
				\draw (0, 0) node[circle,fill,inner sep=1pt,label=above:\hskip1.5em$0$]{};
				\draw (3, 2) node[circle,fill,inner sep=1pt,label=right:$3+2i$]{};
				\draw (2, 3) node[circle,fill,inner sep=1pt,label=right:$2+3i$]{};
				
				\foreach \n in {-3,...,-1,1,2,...,3}{%
					\draw (\n,-3pt) -- (\n,3pt)   node [above] {$\n$};
					\draw (-3pt,\n) -- (3pt,\n)   node [right] {$\n i$};}
			\end{scope}
		\end{tikzpicture}
	\end{center} \caption{изображение суммы $-1+i$ и $3+2i$ на комплексной плоскости}
\end{figure}
Зафиксируем $z_0 \in \mathbb{C} \text{ и } r \in \mathbb{R}, r>0$. Изображение на комплексной плоскости чисел, удовлетворяющих условию: $\left|z-z_0\right|=r$.
\noindent Пусть $z=x+yi $ и $z_0=x_0+y_0i$ . Расписываем модуль разности $\left|z-z_0\right|$ по определению: $\left|z-z_0\right|=\left|x+iy-\left(x_0+iy_0\right)\right|=\left|x-x_0+i\left(y-y_0\right)\right|=\sqrt{\left(x-x_0\right)^2+\left(y-y_0\right)^2}$. Тогда $\left|z-z_0\right|=r\ \leftrightarrow\left(x-x_0\right)^2+\left(y-y_0\right)^2=r^2$ – уравнение окружности.
\begin{exmp}
	$|z-1|+|z+1|=3$
	
	Пусть $z=x+iy$, тогда $\sqrt{(x-1)^2+y^2}+\sqrt{(x+1)^2+y^2}=3$.
	
	\noindent Перенесём второе слагаемое вправо и возведём обе части равенства в квадрат:
	
	\[(x-1)^2+y^2=9-6\sqrt{(x+1)^2+y^2}+(x+1)^2+y^2\]
	
	\noindent или, сокращая: $6\sqrt{(x+1)^2+y^2}=9+4x$.
	
	Возведём обе части равенства в квадрат: $36(x^2+2x+1+y^2)=81+2 \cdot 36x+16x^2$, и далее $20x^2+36y^2=45$.
	
	\noindent Перепишем в виде:
	
	\[\frac{x^2}{a^2}+\frac{y^2}{b^2}=1, \text{где } a^2=\frac{9}{4}, b^2=\frac{5}{4}\]
	
	\noindent Таким образом, исходное уравнение задаёт равносильно каноническому уравнению эллипса.
	
	\begin{figure}[!htbp]
		\begin{center}
			\begin{tikzpicture}
				\draw [->] (-4,0) -- (5,0) node [above left]  {$\realpart z$};
				\draw [->] (0,-4) -- (0,4) node [below right] {$\imgpart z$};
				
				\foreach \n in {-3,...,-1,1,2,...,3}{%
					\draw (\n,-3pt) -- (\n,3pt)   node [above] {$\n$};
					\draw (-3pt,\n) -- (3pt,\n)   node [right] {$\n i$};}
				
				\draw [thick, color=red] (0,0) ellipse (1.5cm and 1cm);
			\end{tikzpicture}
		\end{center}
	\end{figure}
\end{exmp}
\begin{exmp}
	$|z-1|=|z-1|$
	
	Заметим, что $z-i$ - это расстояние от $z$ до $i$, а $|z-1|$ - расстояние от $z$ до $1$. Таким образом, множество точек $z$, удовлетворяющих равенству, является мнодеством точек, равноудалённых от двух данных (от $i$ и от $1$). Это множество представляет собой серединный перпендикуляр к отрезку, соединяющему данные точки:
	
	\begin{figure}[!htbp]
		\begin{center}
			\begin{tikzpicture}
			% Значок перпендикуляра
			
			\draw [->] (-4,0) -- (5,0) node [above left]  {$\realpart z$};
			\draw [->] (0,-4) -- (0,4) node [below right] {$\imgpart z$};
			
			\foreach \n in {-3,...,-1,1,2,...,3}{%
				\draw (\n,-3pt) -- (\n,3pt)   node [above] {$\n$};
				\draw (-3pt,\n) -- (3pt,\n)   node [right] {$\n i$};}
			
			\draw [thick] (0,1) -- (1,0);
			\draw [color=blue, fill=black] (0,1) circle(0.1);
			\draw [color=blue, fill=black] (1,0) circle(0.1);
			\draw [thick, color=red] (-3,-3) -- (3,3);
			
			\end{tikzpicture}
		\end{center}
	\end{figure}	
\end{exmp}
\begin{definition}
	Аргументом комплексного числа $z=a+bi$ называется угол $\phi$ называется угол $\realpart z$ z (оси абсцисс), измеряемый против хода часовой стрелки.
	
	\noindent $Arg\ z=\arg{z+2\pi k,\ k\in Z.\ \pi<\arg{z\ \le\pi}}$ – главное значение аргумента. $Arg\ z$ – совокупность аргументов данного числа.
\end{definition}
\begin{itemize}
	\item Геометрический смысл умножения на мнимую единицу $i$ состоит в повороте на угол $\frac{\pi}{2}$ против часовой стрелки (рис. \ref{complex:graph:rotate}).
	\item Комплексным сопряжением числа $z$ на комплексной плоскости является вектор, симметричный вектору $z$, относительно оси абсцисс (рис. \ref{complex:graph:conjugate}).
	\item Аргумент $\phi$ для 0 не определён.
	\begin{figure}[h]
		\begin{center}
			\begin{tikzpicture}
				\draw [->] (-4,0) -- (5,0) node [above left]  {$\realpart z$};
				\draw [->] (0,-4) -- (0,4) node [below right] {$\imgpart z$};
				
				\draw [->][thick] (0,0) -- (3,3) node [above] {$z_1$};
				\draw [->][thick] (0,0) -- (-4,-2) node [below] {$z_2$};
				
				\draw [<-][red,very thick] (-0.3,-0.1) arc [start angle=255, end angle=5, radius=20pt];
			\end{tikzpicture}
		\end{center} \caption{Умножение на мнимаую единицу} \label{complex:graph:rotate}
	\end{figure}
	
	\begin{figure}[h]
		\begin{center}
			\begin{tikzpicture}
			\draw [->] (-4,0) -- (5,0) node [above left]  {$\realpart z$};
			\draw [->] (0,-4) -- (0,4) node [below right] {$\imgpart z$};
			
			\draw [->][thick] (0,0) -- (3,3) node [above] {$z=x+iy$};
			\draw [->][thick] (0,0) -- (3,-3) node [below] {$\bar z=x-iy$};
			
			\draw[dashed] (3, -3)--(3,3) node[circle,fill,inner sep=1pt]{};
			\draw[dashed] (0, 3)--(3,3) node[circle,fill,inner sep=1pt]{};
			\draw[dashed] (0, -3)--(3,-3) node[circle,fill,inner sep=1pt]{};
			
			\draw (0, 3) node[circle,fill,inner sep=1pt,label=left:$y$]{};
			\draw (0, -3) node[circle,fill,inner sep=1pt,label=left:$-y$]{};
			\draw (0, 0) node[circle,fill,inner sep=1pt,label=below left:$0$]{};
			
			\end{tikzpicture}
		\end{center} \caption{Комплексно сопряжённое число} \label{complex:graph:conjugate}
	\end{figure}
\end{itemize}
\subsection{Тригонометрическое представление комплексного числа, действия над комплексными числами в тригонометрической форме.}
\noindent $x=r\cdot\ cos\phi=\left|z\right|\cos{\phi},\ y=r\cdot\ sin\phi=\left|z\right|\sin{\phi}$, где $r$ - длина радиус вектора, $\phi$ - угол между ним и осью $OX$.
\noindent $z=r(\cos{\left(\phi\right)+i\sin{(\phi))}}$. Где $cos\phi=\frac{x}{\sqrt{x^2+y^2}}$, $sin\phi=\frac{y}{\sqrt{x^2+y^2}}$, $\phi=Arg\ Z,\ r=\left|z\right|=\sqrt{x^2+y^2}$, $\tg \phi=\frac{y}{x}$
\vspace{\baselineskip}
Арифметические операции в тригонометрической форме:
\begin{enumerate}
	\item Умножение:
	\begin{multline}
		z_1 \cdot z_2 = r_1(\cos \phi_1 + i \sin \phi_1) \cdot r_2 (\cos \phi_1 + i \sin \phi_1)= \\ =r_1r_2\left[\cos (\phi_1+\phi_2)+i \sin(\phi_1+\phi_2)\right]
	\end{multline}
	\item Деление:
	\[\frac{z_1}{z_2}= \frac{r_1(\cos \phi_1+i \sin \phi_1)}{r_2(\cos \phi_2+i \sin \phi_2)}=\frac{r_1}{r_2}\left[\cos(\phi_1 - \phi_2)+i \sin (\phi_1 - \phi_2)\right]\]
	\item Сопряжённое и обратное:
	\[\overline{r(\cos \phi + i \sin \phi)}=r(\cos(-\phi)+i \sin (-\phi))\]
	\[\frac{1}{r(\cos \phi + i \sin \phi)}=\frac{1}{r}\left[\cos(-\phi)+i \sin (-\phi)\right]\]
	\item Возведение в степень:
	\[z^n=\left[r(\cos \phi + i \sin \phi)\right]^n=r^n\left[\cos (n\phi)+i \sin (n \phi)\right]\]
	\item Формула Муавра:
	\[(\cos \phi + i \sin \phi)^n=\cos (n \phi)+i \sin(n \phi)\]
	\item Извлечение корня:
	\begin{multline}
	\sqrt[n]{z}=\sqrt[n]{r(\cos \phi + i \sin \phi)}=\sqrt[n]{r}\left(\cos \frac{\phi + 2 \pi k}{n}+i \sin \frac{\phi + 2 \pi k}{n}\right) \\ k=1,...,n-1
	\end{multline}
	\item Условия равенства двух комплексных чисел:
	\begin{enumerate}
		\item В алгебраической форме: $z_1=z_2$, если $\begin{cases}\realpart z_1 = \realpart z_2 \\ \imgpart z_1 = \imgpart z_2\end{cases}$
		\item В тригонометрической форме: $\begin{cases}r_1=r_2 \\ Arg \ z_1 = Arg \ z_2 + 2 \pi n, n \in \mathbb{Z}\end{cases}$
	\end{enumerate}
\end{enumerate}
\subsection{Формула Муавра.}
По формуле произведения двух комплексных чисел в тригонометрической форме получаем, что при произведении комплексного числа в тригонометрической форме самого на себя, его аргумент увеличивается в 2 раза, а модуль умножается сам на себя $\rightarrow$ это верно для возведения в степень:
\[\left(\cos\phi+i\cdot \sin \phi\right)^n=\cos{\left(n\phi\right)}+i\cdot \sin{\left(n\phi\right)}\]
\noindent При $n=0$ формула остаётся верной, так как $\left(\cos \phi+i\cdot \sin \phi\right)^0=cos0+i\cdot \sin 0=1$.
\noindent При $n<1$, т.е. $n=-m: \left(\cos\phi+i\cdot \sin \phi\right)^{-m}=\dfrac{1}{(\cos \phi + i \cdot \sin \phi)^n}$. Далее можно возвести в степень, домножить на сопряжённое и получить итоговый вид, где аргумент умножается на число, обратное $n$.
\subsection{Показательная форма записи комплексного числа, формулы Эйлера.}
\[z=re^{i\varphi},\ \ r=\left|z\right|=\sqrt{x^2+y^2},\ \ \varphi=Arg\ Z\]
Все арифметические операции выполняются по обыкновенным правилам. $z^k=\left(re^{i\phi}\right)^k=r^ke^{ik\phi}$.
Формула Эйлера связывает между собой тригонометрические и показательные функции: $e^{i\phi}=\cos\phi+i \cdot \sin \phi$, где $e$ - экспонента.
Для комлексного числа $z=x+iy$ выполняется: $e^z=e^{x+iy}=e^x \cdot e^{iy}$. В случае, когда $z$ - вещественное число ($\imgpart z = 0$), верно $e^z=e^{x+0i}=e^x \cdot e^0=e^x$. Если же $z$ - чисто мнимое число ($\realpart z = 0$), верно $e^z=e^{0+iy}=e^0 \cdot e^{iy}=e^{iy}$. Используя формулу Эйлера получаем (также см. рис. \ref{complex:graph:euler}):
\[e^z=e^x \cdot e^{iy}=e^x \cdot (\cos y + i \sin y))\]
При помощи формулы Эйлера можно определить функции $\cos$ и $\sin$ следующим образом:
\[\sin \phi=\frac{e^{i\phi}-e^{-i\phi}}{2i},\ \cos \phi=\frac{e^{i\phi}+e^{-i\phi}}{2}\]
\begin{figure}[h]
	\begin{center}
		\begin{tikzpicture}
			\draw [->] (-2,0) -- (2,0) node [above]  {$\realpart z$};
			\draw [->] (0,-2) -- (0,2) node [right] {$\imgpart z$};
			
			\draw[thick,black] (0,0) circle (1cm);
			
			\draw (0, 0) node[circle,fill,inner sep=1pt,label=below left:$0$]{};
			\draw (1, 0) node[circle,fill,inner sep=1pt,label=below right:$1$]{};
			
			\draw [->, very thick] (0,0) -- (0.7,0.7) node [above right]  {\small $e^{i \phi} = \cos \phi + i \sin \phi$};
			
			\draw [red, thick] (0.7,0) -- (0.7,0.7) node [right]  {\footnotesize $\sin \phi$};
			
			\draw [->][red,thick] (0.3,0) arc [start angle=4, end angle=25, radius=20pt] node [left] {\footnotesize \phi};
		\end{tikzpicture}
	\end{center} \caption{Формула Эйлера} \label{complex:graph:euler}
\end{figure}
\subsection{Корни $n$-ой степени из комплексного числа.}
Геометрически все значения корня n-ой степени лежат на окружности радиуса $\sqrt[n]{\left|z\right|}$с центром в начале координат и образуют правильный $n$-угольник. Корень $n$-ой степени $z^\frac{1}{n}$ - это комплексное число $w$, для которого выполнено условие $w^n=z$. Если $z \ne 0$ то существует $n$ различных корней $n$-ой степени из числа $z$.
\[w_k=\sqrt[n]{\left|z\right|} \left(\cos \frac{\phi + 2 \pi k}{n} + i \sin \frac{\phi + 2 \pi k}{n}\right)\]
При изображении корней на комплексной плоскости около точки, с которой отождествляется корень проставляется только его аргумент, поскольку модули у всех корней одинаковые. (у нас есть период, его последовательно и проставляем).
\noindent Кубические корни из единицы: $\left\{1; \frac{-1+i \sqrt{3}}{2}; \frac{-1-i \sqrt{3}}{2}\right\}$. Корни 4-ой степени из единицы: $\left\{1; +i; -1; -i\right\}$. И так далее.
\begin{remark}
	Неравенства для модуля суммы и модуля разности двух комплексных чисел:
	\begin{enumerate}
		\item $|\alpha + \beta| \leq |\alpha| + |\beta|$
		\item $|\alpha - \beta| \leq |\alpha| + |\beta|$
		\item $|\alpha + \beta| \geq |\alpha| - |\beta|$
		\item $|\alpha - \beta| \geq |\alpha| - |\beta|$
	\end{enumerate}
\end{remark}
\section{Матрицы.}
\subsection{Матрицы, основные обозначения, сложение матриц и умножение матрицы на число.}
Матрицей размера $m \times n$ называется таблица, состоящая из $m \times n$ выражений, где $m$ - число строк, $n$ - число столбцов.
\noindent Обозначение:
\[
A=
\begin{pmatrix}
	a_{11} & a_{12} & ... & a_{1n} \\
	a_{21} & a_{22} & ... & a_{2n} \\
	... & ... & ... & ... \\
	a_{m1} & a_{m2} & ... & a_{mn} 
\end{pmatrix}
\text{ или } |A|
\]
\noindent Числа $a_{ij}$ - элементы матрицы. Индекс $i$ обозначает номер строки, а индекс $j$ - столбца.
\begin{enumerate}
	\item Если $m=n$, то матрица квадратная;
	\item Если $m=1$, получаем матрицу-строку $A=\left(a_{11} a_{12} ... a_{1n}\right)$;
	\item Если n=1, получаем матрицу-столбец $A=\begin{pmatrix}a_{11} \\ a_{12} \\ ... \\ a_{m1}\end{pmatrix}$;
	\item Нулевой матрицей (нуль-матрицей) называется матрица любого размера, все элементы которой нули.
\end{enumerate}
\begin{definition}
	Главной диагональю квадратной матрицы называется диагональ, идущая из левого верхнего угла матрицы в ее правый нижний угол: $a_{11}, a_{22},...,a_{nn}$
\end{definition}
\begin{definition}
	Побочной диагональю квадратной матрицы называется диагональ, идущая из левого нижнего угла матрицы в ее правый верхний угол: $a_{n1}, a_{n-12},...,a_{1n}$
\end{definition}
\begin{definition}
	Матрица называется диагональной, если все недиагональные элементы матрицы равны нулю, т.е.
	\[
	a_{ij}=
	\begin{cases}
		a_{ij} \ne 0, i = j \\
		a_{ij} =0, i \ne j 
	\end{cases};
	\begin{pmatrix}
		a_{11} & ... & 0 \\
		0 & a_{22} & 0 \\
		... & ... & ... \\
		0 & ... & a_{nn}
	\end{pmatrix}
	\]
	
	Единичной матрицей $n$-го порядка называется диагональная матрица $n$-го порядка, все диагональные элементы которой равны 1. Обозначают буквой $Е$.
\end{definition}
Две матрицы считаются равными, если эти матрицы имеют одинаковые порядки и все их соответствующие элементы совпадают ($A_{m \times n} = B_{k \times l}, m=k, n=l, a_{ij}=b_{ij}$).
\paragraph{Операции над матрицами:}
\begin{enumerate}
	\item Сложение матриц:
	
	Суммой двух матриц $А$ и $В$ одинакового размера $n \times m$ называется матрица $C=A+B$ того же размера, элементы которой равны $c_{ij}=a_{ij}+b_{ij}, i=1...m, j=1...n$
	\item Умножение матрицы на число:
	
	Произведением матрицы $А$ на вещественное число $\lambda$ называется матрица $B=\lambda A$, элементы которой $b_{ij}=\lambda\ a_{ij}, i=1,…,m; j=1,…,n$
	
	Умножение матрицы на число обладает следующими свойствами:
	\begin{itemize}
		\item $(\lambda\mu)A=\lambda (\mu A)$ - сочетательное относительно числового множителя;
		\item $\lambda(A+B)=\lambda A+\lambda B$ – распределительное относительно суммы матриц;
		\item $(\lambda+\mu) A=\lambda A + \mu A$ - распределительное относительно суммы чисел.
	\end{itemize}
\end{enumerate}
\subsection{Умножение матриц, свойства умножения.}
Умножение матриц: $A_{m\times k}\cdot\ B_{k\times n}=C$, т.е. чтобы умножить матрицу $A$ на матрицу $B$ нужно, чтобы число столбцов матрицы $A$ равнялось числу строк матрицы $B$.
\[\underset{m \times k}{A} \cdot \underset{k \times n}{B} = \underset{m \times n}{C}\]
Каждый элемент матрицы $С$ равен сумме произведений элементов $i$-й строки матрицы $А$ на $j$-й столбец матрицы $В$:
\[c_{ij}=a_{i1}b_{1j}+a_{i2}b_{2j}+...+a_{ik}b_{kj}=\sum\limits_{s=1}^k a_{is}b_{sj}, i=1,2,...,m, j=1,2,...,n\]
\paragraph{Свойства умножения матриц:}
\begin{itemize}
	\item ${AB}C=A(BC)$ - сочетательное свойство, ассоциативность относительно умножения;
	\item $A(B+C)=AB+AC$ - распределительное свойство, дистрибутивность относительно сложения;
	\item $\lambda(AB)=(\lambda A)B$ - сочетательное свойство, ассоциативность относительно умножения на скаляр;
	\item $EA=A; AE=A$ - произведение матрицы на единичную матрицу $E$ подходящего порядка равно самой матрице;
	\item $0A=0; A0=0$ - произведение матрицы на нулевую матрицу $0$ подходящей размерности равно нулевой матрице;
	\item $AB \ne BA$ - умножение матриц в общем случае некоммутативно;
\end{itemize}
\paragraph{Произведение трех матриц:}
\noindent Произведение трёх матриц $ABC$ вычисляют 2-мя способами:
\begin{itemize}
	\item найти $AB$ и умножить на $C: (AB)C$;
	\item либо сначала найти $BC$, а затем уже умножить $A(BC)$.	
\end{itemize}
\textbf{Доказательство некоторых свойств:}
\begin{enumerate}
	\item $(AB)C=A(BC)$ – ассоциативность относительно произведения.
	
	Заметим, что если $A_{m \times n}, B_{n \times p}, C_{p \times s}$, то элемент $d_{ij}$ матрицы $(AB)C$ в силу распределительного св-ва относительно произведения, равен: $d_{ij}=\sum\limits_{k=1}^p\left(\sum\limits_{j=1}^na_{ij}b_{jk}\right) \cdot c_{kl}$, а элемент $d_{ij}'$ матрицы $(AB)C$ равен $d_{ij}'=\sum\limits_{j=1}^n\left(\sum\limits_{k=1}^pb_{jk}c_{kj}\right)$. Тогда равенство $d_{ij}=d_{ij}'$ получаем из возможности изменения порядка суммирования относительно $j$ и $k$.
	\item $(A+B)C=AC+BC$ или $A(B+C)=AB+AC$ - дистрибутивное относительно суммы матриц свойство.
	Если матрицы $A$ и $B$ имеют размер $m \times n$, а матрица $C - n \times k$, то
	\begin{multline}
		\left[(A+B)C\right]_{ij}=\sum\limits_{r=1}^n\left[A+B\right]_{ir}C_{rj}=\sum\limits_{r=1}^n\left[A_{ir}+B_{ir}\right]C_{rj} \hm{=} \\ \sum\limits_{r=1}^n\left[A_{ir}C_{rj}B_{ir}C_{rj}\right] = \sum\limits_{r=1}^nA_{ir}C_{rj}+\sum\limits_{r=1}^nB_{ir}C_{rj}  \hm{=} \\ (AC)_{ij}+(BC)_{ij}=(AC+BC)_{ij}
	\end{multline}
	
	\item Относительно свойства коммутативности произведения матриц отметим следующее:
	\begin{enumerate}
		\item Если произведение матриц $AB$ существует, то произведение $BA$ может и не существовать. Например, если $A_{3 \times 5}, B_{5 \times 4}$, то произведение $AB$ существует, а произведение $BA$ не существует, т.к. число столбцов матрицы $B$ не равно числу строк матрицы $A$.
		\item Если существуют произведения $AB$ и $BA$, то они могут быть матрицами разных размеров. Например, найдём $AB$ и $BA$, если
		\[
		A=
		\begin{pmatrix}
		2 & 4 \\
		-1 & 0 \\
		3 & 1 \\
		\end{pmatrix}, 
		B=
		\begin{pmatrix}
		1 & -1 & 3 \\
		-2 & 5 & 0
		\end{pmatrix}
		\]
		\[
		A_{3 \times 2} \cdot B_{2 \times 3} = C_{3 \times 3} =
		\begin{pmatrix}
		-6 & 18 & 6 \\
		-1 & 1 & -3 \\
		1 & 2 & 9
		\end{pmatrix}
		\]
		\[
		B_{2 \times 3} \cdot A_{3 \times 2} = C_{2 \times 2} =
		\begin{pmatrix}
		12 & 7 \\
		-9 & -8
		\end{pmatrix}, \text{ т.е. } AB \ne BA.
		\]
		\item Если $A$ и $B$ квадратные матрицы одного порядка, то произведения $AB$ и $BA$ существуют и оба являются матрицами одинакового порядка. Но при этом коммутативный (переместительный) закон умножения не выполняется, т.е. всё равно $AB \ne BA$.
		\item Если $D$ – диагональная матрица порядка $n$ такая, что все ее диагональные элементы равны между собой:
		\[
		D=
		\begin{pmatrix}
		d & 0 & 0 & 0 & 0 \\
		0 & d & 0 & 0 & 0 \\
		0 & 0 & d & 0 & 0 \\
		0 & 0 & 0 & d & 0 \\
		0 & 0 & 0 & 0 & d \\
		\end{pmatrix},
		\]
		тогда для $\forall A_{n}$ справедливо равенство $AD=DA$.
		
		Доказательство:
		
		Пусть $c_{ij}$ и $c_{ij}'$ элементы, стоящие на пересечении $i$-й строки и $j$-го столбца матриц $AD$ и $DA$ соответственно. Тогда, $c_{ij}=a_{ij}d$, $c_{ij}'=da_{ij}$, т.е. $c_{ij}=c_{ij}'$.
	\end{enumerate}
\end{enumerate}
\subsection{Элементарные преобразования матриц, матрицы элементарных преобразований их связь.}
\noindent Элементарными преобразованиями строк называют:
\begin{itemize}
	\item перестановку местами любых двух строк матрицы, при этом определитель матрицы меняет знак;
	\item умножение любой строки матрицы на константу $\lambda \ne 0$, при этом определитель матрицы увеличивается в $\lambda$ раз;
	\item прибавление к любой строке матрицы другой строки, умноженной на некоторую константу.
\end{itemize}
Аналогично определяются элементарные преобразования столбцов.
Матрицы $A$ и $B$ называют эквивалентными матрицами если от матрицы $A$ к матрице $B$ перешли с помощью элементарных преобразований над строками и обозначают $A \sim B$.
Система линейных алгебраических уравнений (СЛАУ), полученная путём элементарных преобразований над исходной системой, эквивалентна ей.
\paragraph{Матрица элементарных преобразований:}
\begin{definition}
	Квадратная матрица, получающаяся из единичной матрицы в результате неособенного элементарного преобразования над строками (столбцами), называется элементарной, соответствующей этому преобразованию.
\end{definition}
\noindent Например: $A^{-1}=M_{12}(-1)M_{21}(-3)$. Объяснение последней матрицы: складываем вторую и первую с коэффициентом  ($-3$). Первое преобразование матрицы в записи будет последним.
\[M_{12}(-1)M_{21}(-3)(A|E)=(E|A)^{-1}\]
\paragraph{Связь элементарных преобразований и матриц элементарных преобразований:}
Пусть дана матрица $A$ размером $m \times n$.
\begin{itemize}
	\item Перестановка двух столбцов (строк) матрицы.
	
	Для перестановки двух столбцов ($i$-гo и $j$-го) данной матрицы достаточно умножить её справа на квадратную матрицу $M$ $n$-го порядка, полученую из единичной матрицы $n$-го порядка при помощи перестановки $i$-гo и $j$-го столбцов. Чтобы поменять местами две строки ($i$-ю и $j$-ю) данной матрицы, достаточно умножить ее слева на элементарную матрицу, полученную из единичной матрицы $m$-го порядка при помощи перестановки $i$-й и $j$-й строк.
	\item Умножение всех элементов одного столбца (строки) матрицы на одно и то же число, отличное от нуля.
	
	Для умножения всех элементов одного столбца ($i$-гo) данной матрицы на одно и то же число $\lambda$, отличное от нуля, достаточно умножить матрицу $A$ справа на элементарную матрицу, полученную из единичной матрицы $n$-го порядка умножением $i$-го столбца на число $\lambda$.
	
	Чтобы умножить все элементы $i$-й строки данной матрицы на одно и то же число $\lambda$, отличное от нуля, достаточно умножить матрицу $A$ слева на элементарную матрицу, полученную из единичной матрицы $m$-го порядка умножением $i$-й строки на число $\lambda$.
	\item Прибавление к элементам одного столбца (строки) соответствующих элементов другого столбца (строки), умноженных на одно и то же число.
	
	Чтобы прибавить к одному столбцу ($i$-му) соответствующие элементы другого столбца ($j$-го), умноженные на одно и то же число $\lambda$, достаточно умножить матрицу $A$ справа на элементарную матрицу, полученную из единичной матрицы $n$-го порядка в результате прибавления к $i$-му столбцу соответствующих элементов $j$-го столбца, умноженных на число $\lambda$.
	
	Чтобы прибавить к одной строке ($i$-й) соответствующие элементы другой строки ($j$-й), умноженные на одно и то же число $\lambda$, достаточно умножить матрицу $A$ слева на элементарную матрицу, полученную из единичной матрицы $m$-го порядка прибавлением к элементам $i$-й строки соответствующих элементов $j$-й строки, умноженных на число $\lambda$.
\end{itemize}
В матрице элементарных преобразований в нижних индексах: при преобразовании строк: первое – что меняем, второе – за счёт чего; при преобразовании столбцов: первое – за счёт чего меняем, второе – что меняем.
Элементарные преобразования обратимы, а обратные им преобразования являются элементарными преобразованиями того же самого типа, т.е. если матрица $B$ получена из матрицы $A$ с помощью элементарного преобразования, тогда матрица $A$ может быть получена из матрицы $B$ с помощью элементарного преобразования того же самого типа.
\subsection{Ступенчатые матрицы, приведение матрицы к ступенчатому виду с помощью элементарных преобразований.}
Ступенчатая матрица — матрица, имеющая $m$ строк, у которой первые $r$ диагональных элементов ненулевые, $r \le m$, а элементы, лежащие ниже диагонали и элементы последних $m-r$ строк равны нулю (если $a_{1;k_1}, a_{2;k_2},...,a_{r;k_r}$ - ведущие элементы ненулевых строк матрицы, то $k_1 < k_2 < ... < k_r$.
\[
A=
\begin{pmatrix}
a_{11} & a_{12} & a_{13} & ... & a_{1r} & ... & a_{1n} \\
0 & a_{22} & a_{23} & ... & a_{2r} & ... & a_{2n} \\
0 & 0 & a_{33} & ... & a_{3r} & ... & a_{3n} \\
... & ... & ... & ... & ... & ... & ... \\
0 & 0 & 0 & ... & a_{rr} & ... & a_{rn} \\
0 & 0 & 0 & ... & 0 & ... & 0 \\
... & ... & ... & ... & ... & ... & ... \\
0 & 0 & 0 & ... & 0 & ... & 0 \\
\end{pmatrix}, r \le min(m, n).
\]
\noindent Алгоритм приведения матрицы к ступенчатому виду:
\begin{enumerate}
	\item В первом столбце выбрать элемент, отличный от нуля (ведущий элемент). Строку с ведущим элементом (ведущая строка), если она не первая, переставить на место первой строки (преобразование I типа). Если в первом столбце нет ведущего (все элементы равны нулю), то исключаем этот столбец, и продолжаем поиск ведущего элемента в оставшейся части матрицы. Преобразования заканчиваются, если исключены все столбцы или в оставшейся части матрицы все элементы нулевые.
	\item Разделить все элементы ведущей строки на ведущий элемент (преобразование II типа). Если ведущая строка последняя, то на этом преобразования следует закончить.
	\item К каждой строке, расположенной ниже ведущей, прибавить ведущую строку, умноженную соответственно на такое число, чтобы элементы, стоящие под ведущим оказались равными нулю (преобразование III типа).
	\item Исключив из рассмотрения строку и столбец, на пересечении которых стоит ведущий элемент, перейти к пункту 1, в котором все описанные действия применяются к оставшейся части матрицы.
\end{enumerate}
\subsection{Обратная матрица, вычисление обратной матрицы с помощью элементарных преобразований. Метод Гаусса (прямой ход).}
\begin{enumerate}
	\item Составляем блочную матрицу $A|E$, приписав к данной матрице $A$ справа единичную матрицу того же порядка.
	\item При помощи элементарных преобразований, выполняемыми над строками матрицы ($A|E$) приводим её левую часть к простейшему виду:
	\[
	\begin{pmatrix}
	E & 0 \\
	0 & 0
	\end{pmatrix}
	\]
	При этом блочная матрица приводится к виду $\lambda|S$, где $S$ - квадратная матрица, полученная в рузультате преобразования из единичной матрицы $E$.
	\item Если $\lambda=E$, то блок $S=A^{-1}$, если $\lambda \ne E$, то матрица $A$ - не имеет обратной, т.е. она вырожденная.
\end{enumerate}
\begin{theorem}
	Квадратная матрица $A$ обратима (имеет обратную матрицу) тогда и только тогда, когда она невырожденная, то есть $\det A \ne 0$.
\end{theorem}
\begin{evidence}
	Если матрица $A$ обратима, то $AB=E$ для некоторой матрицы $B$. Тогда, если квадратные матрицы одного и того же порядка, то $\det AB = \det A \cdot \det B$:
	
	\noindent $1=\det E = \det AB = \det A \cdot \det B$, следовательно, $\det A \ne 0, \det B \ne 0$.
\end{evidence}
\begin{remark}
	\indent
	\begin{enumerate}
		\item Пусть $A \in M_{n}(\mathbb{R}), (A|E)$ - расширенная матрица. Тогда если в результате применения к расширенной матрице элементарных преобразований, получена строго треугольная матрица с ненулевой диагональю $A|E \rightarrow (T|B), \text{ где } t_{i,j} \ne \forall i$, тогда матрица $A$ обратима (т.е. $\exists A^{-1}$) и обратная матрица получается в правой половине расширенной матрицы $(T|B) \rightarrow (E|A^{-1})$ если в левой половине получить единичную матрицу.
		\item Если в результате прямого хода метода Гаусса к матрице $A$ получается матрица с нулевой строкой, то матрица $A$ – необратима.
	\end{enumerate}
\end{remark}
\subsection{Подстановки и их простейшие свойства. Перестановки.}
\begin{definition}
	Всякое расположение чисел $1, 2, 3,...,n$ в некотором порядке называется перестановкой из $n$ чисел. Общий вид записи перестановки из $n$ элементов: $i_1, i_2,...,i_n$, где каждое $i_s$ есть одно из чисел $1, 2, 3,...,n$, причем ни одно из этих чисел не встречается дважды и не пропущено. Число перестановок из $n$ символов равно $n!$.
\end{definition}
Если в некоторой перестановке поменяем местами какие-либо два символа (не обязательно стоящие рядом), а все остальные оставим на месте, то получим новую перестановку. Такое преобразование перестановки называется \textit{транспозицией}. Любая транспозиция меняет чётность.
\begin{definition}
	Инверсией в перестановке $\pi$ порядка $n$ называется всякая пара индексов $i, j$ такая, что $1 \le i \le j \le n$ и $\pi (i) > \pi (j)$. Чётность числа инверсий в перестановке определяет чётность перестановки.
\end{definition}
Перестановка с естественным порядком – \textit{тривиальная}. От неё к чётной можно перейти за чётное число транспозиций, а к нечётной – за нечётное.
\begin{exmp}
	Определить число инферсий в перестановке: $5, 2, 1, 4, 3$.
	
	\noindent Имеем последовательно:
	\begin{equation}
		\begin{aligned}
			& 5, 2, 1, 4, 3 & - 2 \text{ числа перед } 1; \\
			& 5, 2, 4, 3 & - 1 \text{ число перед } 2; \\
			& 5, 4, 3 & - 2 \text{ числа перед } 3; \\
			& 5, 4 & - 1 \text{ число перед } 4; \\
			& 5 & - 0 \text{ чисел перед } 5.
		\end{aligned}
	\end{equation}
	Таким образом, $N(5, 2, 1, 4, 3) = 2 + 1 + 2 + 1 + 0 = 6$.
\end{exmp}
\begin{definition}
	Запишем одну перестановку под другой:
	\begin{equation}\label{eq:perm}
	\begin{pmatrix}
	i_ 1 & i_2 & i_3 & ... & i_n \\
	\alpha_{i_1} & \alpha_{i_2} & \alpha_{i_3} & ... & \alpha_{i_n}
	\end{pmatrix}.
	\end{equation}
	Эту запись называют подстановкой, понимая под этим отображение (соответствие) множества символов, состоящего из первых $n$ чисел: $1, 2, 3,...,n$ на себя:
	\[i_1 \rightarrow \alpha_{i_1}, i_2 \rightarrow \alpha_{i_2}, i_3 \rightarrow \alpha_{i_3}, ..., i_n \rightarrow \alpha_{i_n}\]
\end{definition}
Если учесть, что подстановка как отображение множества чисел $1, 2, 3,...,n$ не меняется при транспозиции столбцов, выберем для нее простейшее выражение:
\[
\begin{pmatrix}
1 & 2 & 3 & ... & n \\
\alpha_{1} & \alpha_{2} & \alpha_{3} & ... & \alpha_{4}
\end{pmatrix},
\] где $\alpha_i$ - число, в которое переходит число $i$.
В выражении подстановки порядка n различаются только перестановками в нижней строке записи, т.е. подстановку однозначно определяет перестановка, записанная в ее нижней строке. Это значит, что всего подстановок порядка $n$ столько же, сколько и перестановок, т.е. $n!$.
\begin{exmp}
	Запишем друг под другом две перестановки и рассмотрим подстановку 5-ой степени.
	
	\begin{equation}\label{eq:perm2}
	\begin{pmatrix}
	3 & 5 & 1 & 4 & 2 \\
	5 & 2 & 3 & 4 & 1
	\end{pmatrix}
	\end{equation}
	В данном случае число 3 переходит в 5, число 5 переходит в 2, число 1 переходит в 3, число 4 переходит в 4 (или остаётся на месте) и, наконец, число 2 переходит в 1. Таким образом, две перестановки, записанные друго под другом в виде \eqref{eq:perm2}, определяют некоторое взаимное однозначное отображение множества из первых пяти натуральных чисел самих на себя, т.е. отображение, которое каждому из натуральных чисел $1, 2, 3, 4, 5$ ставит в соответствие одно из этих же натуральных чисел, причем разным числам ставятся в соответствие различные же числа. При этом, так как чисел всего пять, т.е. конечное множество, каждое из этих пяти чисел будет соответствовать одному из чисел $1, 2, 3, 4, 5$, а именно числу, которое в него «переходит».
	
	Ясно, что то взаимно однозначное отображение множества из первых пяти натуральных чисел, которое мы получили при помощи \eqref{eq:perm2}, можно было бы получить, записывая одну под другой и некоторые другие пары перестановок из пяти символов. Эти записи получаются из \eqref{eq:perm2} путем нескольких транспозиций столбиков; таковы, например,
	\begin{equation}\label{eq:perm3}
	\begin{pmatrix}
	2 & 1 & 5 & 3 & 4 \\
	1 & 3 & 2 & 5 & 4
	\end{pmatrix},
	\begin{pmatrix}
	1 & 5 & 2 & 4 & 3 \\
	3 & 2 & 1 & 4 & 5
	\end{pmatrix},
	\begin{pmatrix}
	2 & 5 & 1 & 4 & 3 \\
	1 & 2 & 3 & 4 & 5
	\end{pmatrix}
	\end{equation}
	Во всех этих записях 3 переходит в 5, 5 в 2, и т.д.
	
	Аналогичным путем две перестановки из 7 символов, записанные одна под другой, определяют некоторое, взаимно однозначное отображение множества первых $n$ натуральных чисел на себя. Всякое взанмно однозначное отображение $A$ множества первых $n$ натуральных чисел на себя называется подстановкой $n$-й степени, причем, очевидно, всякая подстановка А может быть записана при помощи двух перестановок, подписанных одна под другой в виде \eqref{eq:perm}.
\end{exmp}
\paragraph{Определим понятие четности для подстановок:}
\begin{itemize}
	\item Исходя из общего определения подстановки:
		\begin{itemize}
			\item подстановка четная, если четности верхней и нижней перестановок совпадают;
			\item подстановка нечетная, если четности верхней и нижней перестановок противоположны.
		\end{itemize}
	\item Учитывая частную запись подстановки:
		\begin{itemize}
			\item подстановка четная, если ее определяет четная перестановок;
			\item подстановка нечетная, если ее определяет нечетная перестановка.
		\end{itemize}
\end{itemize}
Кроме подсчета числа инверсий в перестановках для определения четности подстановок применяют также разложение их в циклы.
Некоторые свойства:
\begin{itemize}
	\item Число чётных подстановок $n$-й степени равно числу нечётных, т.е. $n!$;
	\item Подстановка $A$ будет чётной, если общее число инверсий в двух строках любой её записи чётно, и нечётной – в противоположном случае;
	\item Произведение любой подстановки $A$ на тождественную подстановку $E$, а также произведение $E$ на $A$, равно $A$.
\end{itemize}
\subparagraph{Определение определителя, его простейшие свойства (полилинейность, поведение определителя при элементарных преобразованиях, определитель с нулевой строкой и двумя одинаковыми строками).}
Определитель матрицы является многочленом от элементов квадратной матрицы (т.е. такой, у которой количество строк и столбцов равны). В общем случае матрица может быть определена над любым коммутативным кольцом, в этом случае определитель будет элементом того же кольца.
\paragraph{Правило треугольника для вычисления определителя:}
\begin{figure}[!h]
	\begin{center}
		\begin{tikzpicture}
		\draw [-] (-0.5,-0.5) -- (-0.5,2.5) node [left] {$,, + ``$};
		
		\draw (0, 0) node[circle,fill,inner sep=2pt]{};
		\draw (0, 1) node[circle,fill,inner sep=2pt]{};
		\draw (0, 2) node[circle,fill,inner sep=2pt]{};
		\draw (1, 0) node[circle,fill,inner sep=2pt]{};
		\draw (1, 1) node[circle,fill,inner sep=2pt]{};
		\draw (1, 2) node[circle,fill,inner sep=2pt]{};
		\draw (2, 0) node[circle,fill,inner sep=2pt]{};
		\draw (2, 1) node[circle,fill,inner sep=2pt]{};
		\draw (2, 2) node[circle,fill,inner sep=2pt]{};
		
		\draw [-, thick] (0,0) -- (1,2);
		\draw [-, thick] (0,0) -- (2,1);
		\draw [-, thick] (1,2) -- (2,1);
		
		\draw [-, thick] (1,0) -- (0,1);
		\draw [-, thick] (1,0) -- (2,2);
		\draw [-, thick] (0,1) -- (2,2);
		
		\draw [-, thick] (0,2) -- (2,0);
		
		\draw [-] (2.5,-0.5) -- (2.5,2.5);
		
		
		\draw [-] (3,-0.5) -- (3,2.5);
		
		\draw (4, 0) node[circle,fill,inner sep=2pt]{};
		\draw (4, 1) node[circle,fill,inner sep=2pt]{};
		\draw (4, 2) node[circle,fill,inner sep=2pt]{};
		\draw (5, 0) node[circle,fill,inner sep=2pt]{};
		\draw (5, 1) node[circle,fill,inner sep=2pt]{};
		\draw (5, 2) node[circle,fill,inner sep=2pt]{};
		\draw (6, 0) node[circle,fill,inner sep=2pt]{};
		\draw (6, 1) node[circle,fill,inner sep=2pt]{};
		\draw (6, 2) node[circle,fill,inner sep=2pt]{};
		
		\draw [-, thick] (4,2) -- (5,0);
		\draw [-, thick] (4,2) -- (6,1);
		\draw [-, thick] (5,0) -- (6,1);
		
		\draw [-, thick] (6,0) -- (4,1);
		\draw [-, thick] (6,0) -- (5,2);
		\draw [-, thick] (4,1) -- (5,2);
		
		\draw [-, thick] (4,0) -- (6,2);
		
		\draw [-] (6.5,-0.5) -- (6.5,2.5) node [right] {$,, - ``$};
		
		\end{tikzpicture}
	\end{center} \caption{Правило треугольника для матрицы $3 \times 3$} \label{matrix:fig:mult3}
\end{figure}
\begin{figure}[!h]
	\begin{center}
		\begin{tikzpicture}
			\draw (0, 0) node[circle,fill,inner sep=2pt]{};
			\draw (0, 1) node[circle,fill,inner sep=2pt]{} node [left] {$,, + ``$};
			
			\draw (1, 0) node[circle,fill,inner sep=2pt]{} node [right] {$a_{21}$};
			\draw (1, 1) node[circle,fill,inner sep=2pt]{} node [right] {$a_{11}$};
			
			\draw (3, 0) node[circle,fill,inner sep=2pt]{} node [left] {$a_{22}$};
			\draw (3, 1) node[circle,fill,inner sep=2pt]{} node [left] {$a_{12}$};
			
			\draw (4, 0) node[circle,fill,inner sep=2pt]{} ;
			\draw (4, 1) node[circle,fill,inner sep=2pt]{} node [right] {$,, - ``$};
			\draw [-, thick] (0,1) -- (1,0);
			
			\draw [-, thick] (3,0) -- (4,1);
		\end{tikzpicture}
	\end{center} \caption{Правило треугольника для матрицы $2 \times 2$} \label{matrix:fig:mult2}
\end{figure}
Произведение элементов в первом определителе, которые соединены прямыми, берется со знаком "плюс"; аналогично, для второго определителя - соответствующие произведения берутся со знаком "минус", т.е. используем правило \ref{matrix:fig:mult2} и \ref{matrix:fig:mult3}.
Определитель можно найти разложив строку или столбец матрицы на сумму её/его алгебраических дополнений.
\paragraph{Свойства определителя:}
\begin{itemize}
	\item Определитель — полилинейная функция строк (столбцов) матрицы. Полилинейность означает, что определитель линеен по каждой строке (по каждому столбцу).
	\item Определитель — кососимметрическая функция строк (столбцов) матрицы (другими словами, если переставить две строки (столбца) матрицы, то её определитель умножается на (-1)).
	\item Если какая–либо строка (или столбец) определителя состоит из одних нулей, то такой определитель равен нулю.
	\item При транспонировании матрицы ее определитель не изменяется.
	\item При перестановке местами двух строк матрицы ее определитель сохраняет свою абсолютную величину, но меняет знак на противоположный.
	\item Определитель, имеющий две одинаковые строки, равен нулю.
	\item Умножение всех элементов некоторой строки определителя на число $\lambda$ равносильно умножению определителя на $\lambda$.
	\item Если элементы двух строк определителя пропорциональны, то определитель равен нулю.
	\item Определитель не изменится, если к элементам некоторой его строки прибавить соответствующие элементы любой другой строки, умноженные на произвольное число $\lambda$.
	\item Определитель матрицы, содержащий нулевую строку или столбец, равен нулю.
	\item Определитель верхней или нижней треугольной матрицы равен произведению его диагональных элементов.
\end{itemize}
\subsection{Определение определителя через перестановки.}
Определителем квадратной матрицы $n$-го порядка, или определителем $n$-го порядка, называется число, равное алгебраической сумме $n!$ членов, каждый из которых составлен следующим образом. Каждое слагаемое — это произведение $n$ элементов матрицы, взятых по одному из каждой строки и каждого столбца, умноженное на $-1$ в степени количества инверсий: $|A|=\sum\limits_{n!}(-1)^{N(p)} \cdot a_{1;p_1} \cdot a_{2;p_2} \cdot ... \cdot a_{n;p_n}$, где $p$ - перестановка, $N(p)$ - количество инверсий в перестановке $p$.
Запись через $\delta$-функцию Дирака: $|A| = \sum_{\delta}(-1)^{\delta} \cdot a_{1;\delta(1)} \cdot a_{2;\delta(2)} \cdot ... \cdot a_{n;\delta(n)}$, где $(-1)^{\delta}=\begin{cases}&1, \text{ если } \delta \text{ - чёт.} \\ &-1, \text{ если } \delta \text{ - нечёт.}\end{cases}$
\begin{evidence}
	Пусть имеется матрица
	
	\begin{multline}\label{eq:determsum}
	|A|=\begin{vmatrix}a_{11} & 0 & ... & 0 \\ * & * & * & *\end{vmatrix} + \begin{vmatrix}0 & a_{12} & ... & 0 \\ * & * & * & *\end{vmatrix} + ... + \begin{vmatrix}0 & 0 & ... & a_{1n} \\ * & * & * & *\end{vmatrix} + ... \hm{=} \\ ... = n^n
	\end{multline} слагаемых (далее то же самое со второй строчкой). В кажой строке ровно один элемент! Если хотя бы два одинаковых, определитель равен 0.
\end{evidence}
\begin{property}
	Если в определителе переставить местами любые две строки или два столбца, то определитель изменяет свой знак на противоположный.
	\begin{evidence}
		Т.к. определитель транспонированной матрицы равен определителю исходной матрицы, то любая транспозиция изменяет четность перестановки. Следовательно, при перестановке двух строк (столбцов) каждое слагаемое суммы \eqref{eq:determsum} изменяет свой знак на противоположный.
	\end{evidence}
\end{property}
\begin{property}
	Если матрица содержит нулевую строку (столбец), то определитель этой матрицы равен нулю.
	\begin{evidence}
		Каждая строка и каждый столбец матрицы $A$ представлены одним из своих элементов в произведении $a_{1k_1}, a_{2k_2}, ..., a_{nk_n}$. Следовательно, сумма \eqref{eq:determsum} содержит только нулевые слагаемые.
	\end{evidence}
\end{property}
\begin{property}
	Если две строки (столбца) матрицы равны между собой, то определитель этой матрицы равен нулю.
	\begin{evidence}
		По cвойству 1, при перестановке двух строк местами определитель изменяет свой знак. С другой стороны, перестановка местами одинаковых строк не изменяет определитель. Следовательно, $\det A = - \det A$, что влечет $\det A = 0$.
	\end{evidence}
\end{property}
\begin{property}
	Если две строки (столбца) матрицы пропорциональны друг другу, то определитель этой матрицы равен нулю.
	\begin{evidence}
		Общий множитель строки можно вынести за знак определителя. Полученный при этом определитель имеет две одинаковых строки. Согласно 3-му cвойству такой определитель равен нулю.
	\end{evidence}
\end{property}
\subsection{Определитель транспонированной матрицы.}
Определитель транспонированной матрицы равен определителю исходной матрицы: $\det A^T = \det A$.
\begin{evidence}
	При транспозиции сумма остаётся прежней, только меняется упорядоченность (не по строкам, а по столбцам):
	\[
	|A| = \sum_{\delta}(-1)^{\delta} \cdot a_{1;\delta(1)} \cdot a_{2;\delta(2)} \cdot ... \cdot a_{n;\delta(n)} = \sum_{\delta}(-1)^{\delta} \cdot a_{1;\delta'(1)} \cdot a_{2;\delta'(2)} \cdot ... \cdot a_{n;\delta'(n)}
	\]
	вернулись обратно за то же число инверсия $\rightarrow$ чётность равна. 
	
	\textit{Вывод:} при транспонировании матрицы $A$ происходит лишь перегруппировка слагаемых в этой сумме.
\end{evidence}
\subsection{Теорема о разложении определителя по элементам строки или столбца.}
\textbf{По элементам строки:}
\noindent Определитель матрицы $A$ равен сумме произведений элементов строки на их алгебраические дополнения.
\[|A|=a_{i1}A_{i1}+...+a_{in}A_{in}\left(=\sum\limits_{j=1}^na_{ij}A_{ij}\right)\]
\textbf{По элементам столбца:}
\noindent Определитель матрицы $A$ равен сумме произведений элементов столбца на их алгебраические дополнения.
\[|A|=a_{1j}A_{1j}+...+a_{nj}A_{nj}\left(=\sum\limits_{i=1}^na_{ij}A_{ij}\right)\]
\begin{remark}
	Рассмотрим квадратную матрицу $A$ $n$-го порядка.   Выберем  $i,j$-ый элемент этой матрицы и вычеркнем $i$-ую строку и $j$-ый столбец. В результате мы получаем матрицу $(n-1)$-го порядка, определитель которой называется минором элемента и обозначается символом $M_{ij}$:
	\[
	\begin{pmatrix}
		a_{11} & ... & a_{1j} & ... & a_{1n} \\
		... & ... & ... & ... & ... \\
		a_{i1} & ... & \textcolor{red} {a_{ij}} & ... & a_{1n} \\
		... & ... & ... & ... & ... \\
		a_{n1} & ... & a_{n2} & ... & a_{nn}
	\end{pmatrix}, A_{ij} = (-1)^{i+j}M_{ij}
	\]
	\begin{multline}
	A_{ij} = \sum\limits_{\{k_1,...,k_{i-1},k_i-j,k_{i+1},...,k_n\}} a_{1k_1}a_{2k_2}...a_{i-1,k_{i-1}}a_{i+1,k_{i+1}}...a_{nk_n} \hm{\cdot} \\ (-1)^{P\{k_1,...,k_{i-1},k_i-j,k_{i+1},...,k_n\}}
	\end{multline}
\end{remark}
\begin{lemma}
	Если в матрице $A=||a_{ij}||$ все элементы $i$-ой строки ($j$-ого столбца), кроме одного элемента $a_{ij}$, равны нулю, то определитель матрицы равен произведению этого элемента на его алгебраическое дополнение, т.е. $|A|=a_{ij}A_{ij}$.
	\begin{evidence}
		\[
		A=
		\begin{vmatrix}
		a_{11} & ... & a_{1k-1} & a_{1k} & a_{1k+1} & ... & a_{1n} \\
		... & ... & ... & ... & ... & ... & ... \\
		0 & ... & 0 & a_{ik} & 0 & ... & 0 \\
		... & ... & ... & ... & ... & ... & ... \\
		a_{n1} & ... & a_{nk-1} & a_{nk} & a_{nk+1} & ... & a_{nn} \\
		\end{vmatrix} = a_{ik}A_{ik}
		\]
		Переставляя последовательно $i$-ю строку $(i-1)$ раз с $(i-1)$ строками, стоящими над ней, а затем переставляя последовательно $k$-й столбец $(k-1)$ раз с $(k-1)$ столбцами, стоящими левее его, получаем:
		\begin{multline}
		|A|=(-1)^{(i-1)+(k-1)}
		\begin{vmatrix}
		a_{ik} & 0 & ... & 0 & 0 & ... & 0 \\
		... & ... & ... & ... & ... & ... & ... \\
		a_{1k} & a_{11} & ... & a_{1k-1} & a_{1k+1} & ... & a_{1n} \\
		... & ... & ... & ... & ... & ... & ... \\
		a_{nk} & a_{n1} & ... & a_{nk-1} & a_{nk+1} & ... & a_{nn} \\
		\end{vmatrix} \hm{=} \\ (-1)^{i+k}a_{ik}M_{ik} = a_{ik}A_{ik}
		\end{multline}
	\end{evidence}
\end{lemma}
\subsection{Линейная зависимость и независимость строк (столбцов) матрицы, свойства.}
\begin{definition}\label{def:matrix:lnz}
	Строки (столбцы) $a_1, ..., a_s$ называют линейно независимыми (ЛНЗ), если равенство $\alpha_1 a_1 + ... + \alpha_s a_s = 0$, где 0 в правой части — нулевая строка (столбец), возможно лишь при $\alpha_1=...=\alpha_s=0$. В противном случае, когда существуют такие действительные числа $\alpha_1,...,\alpha_s$ , не равные
	нулю одновременно, что выполняется указанное равенство, эти строки (столбцы) называют линейно зависимыми (ЛЗ).
\end{definition}
Следующее утверждение известно как критерий линейной зависимости.
\begin{theorem}\label{th:matrix:lnz}
	Строки (столбцы) $a_1, ..., a_s, s > 1$, линейно зависимы тогда и только тогда, когда хотя бы одна (один) из них является линейной комбинацией остальных.
	\begin{evidence}
		Доказательство проведем для строк, а для столбцов оно аналогично.
		
		\noindent \textit{Необходимость}. Если строки $a_1, ..., a_s$ линейно зависимы, то, согласно определению \ref{def:matrix:lnz}, существуют такие действительные числа $\alpha_1,...,\alpha_s$, не равные нулю одновременно, что $\alpha_1 a_1 + ... + \alpha_s a_s = 0$. Выберем ненулевой коэффициент $\alpha a_i$. Для определённости пусть это будет $\alpha_1$. Тогда $\alpha_1a_1=(-\alpha_2)a_2+...+(-\alpha_s)a_s$ и, следовательно, $a_1=(-\frac{\alpha_2}{\alpha_1})a_2+...+(-\frac{\alpha_s}{\alpha_1})a_s$, т.е. строка $a_1$ представляется в виде линейной комбинации остальных строк.
		
		\noindent \textit{Достаточность}. Пусть, например, $a_1=\lambda_2a_2+...+\lambda_sa_s$. Тогда $1a_1\hm{+}(-\lambda_2)a_2+...+(\lambda_s)a_s=0$ Первый коэффициент линейной комбинации равен единице, т.е. он ненулевой. Согласно определению \ref{def:matrix:lnz}, строки $a_1, ..., a_s$ линейно зависимы.
	\end{evidence}
\end{theorem}
\begin{theorem}
	Пусть строки (столбцы) $a_1, ..., a_s$ линейно независимы, а хотя бы одна из строк (столбцов) $b_1, ..., b_l$ является их линейной комбинацией. Тогда все строки (столбцы) $a_1, ..., a_s, b_1, ..., b_l$ линейно зависимы.
	\begin{evidence}
		Пусть, например, $b_1$ есть линейная комбинация $a_1, ..., a_s$, т.е. $b_1 = \alpha_1 a_1 + ... + \alpha_s a_s, a_i \in \mathbb{R}, i = 1,...,s$. В эту линейную комбинацию добавим строки (столбцы) $b_2, ..., b_l$ (при $l>1$) с нелувыми коэффициентами: $b_1 = \alpha_1 a_1 + ... + \alpha_s a_s + 0b_2 + ... + 0b_l$. Согласно теореме \ref{th:matrix:lnz}, строки (столбцы) $a_1, ..., a_s, b_1, ..., b_l$ линейной зависимы.
	\end{evidence}
\end{theorem}
Понятия линейной зависимости и линейной независимости определяются для строк и столбцов одинаково. Поэтому свойства, связанные с этими понятиями, сформулированные для столбцов, разумеется, справедливы и для строк.
\begin{enumerate}
	\item Если в систему столбцов входит нулевой столбец, то она линейно зависима.
	\item Если в системе столбцов имеется два равных столбца, то она линейно зависима.
	\item Если в системе столбцов имеется два пропорциональных столбца ($A_i = \lambda A_j$), то она линейно зависима.
	\item Система из $k>1$ столбцов линейно зависима тогда и только тогда, когда хотя бы один из столбцов есть линейная комбинация остальных.
	\item Любые столбцы, входящие в линейно независимую систему, образуют линейно независимую подсистему.
	\item Система столбцов, содержащая линейно зависимую подсистему, линейно зависима.
	\item Если система столбцов $A_1, A_2, ..., A_k$ - линейно независима, а после присоединения к ней столбца $A$ оказывается линейно зависимой, то столбец $A$ можно разложить по столбцам $A_1, A_2, ..., A_k$, и притом единственным образом, т.е. коэффициенты разложения находятся однозначно.
\end{enumerate}
\subsection{Ранг матрицы, его свойства.}
\begin{definition}
	Ранг матрицы – это максимальное количество линейно независимых строк. Или: ранг матрицы – это максимальное количество линейно независимых столбцов.
	
	Рангом матрицы называется наивысший порядок отличных от нуля миноров этой матрицы.
\end{definition}
\textbf{Свойства:}
\begin{enumerate}
	\item Ранг произвольной матрицы $A$ не превосходит наименьший из ее размеров: $r(A) \le min\{m, n\}$.
	\item Если все элементы матрицы $A$ равны нулю, то ранг такой матрицы равен нулю.
	\item Если $A$ - невырожденная квадратная матрица $n$–го порядка, то ее ранг совпадает с порядком матрицы: $r = n$.
	\item Ранг матрицы не изменится, если к ее строкам (столбцам) применить элементарные преобразования.
	\item Ранг ступенчатой матрицы равен количеству её ненулевых строк.
	\item Для квадратной матрицы $n$-ого порядка $r=n$ только тогда, когда матрица невырожденная (определитель отличен от нуля).
\end{enumerate}
\subsection{Ранг стандартной матрицы.}
\begin{definition}
	Ранг матрицы $A$ - максимальный порядок неравного нулю минора (минор - определитель квадратной матрицы $k \times k, k \le n$). Обозначается $r_A$.
\end{definition}
\begin{definition}
	Минор, определяющий ранг матрицы, называется базисным минором. Строки и столбцы, формирующие БМ, называются базисными строками с столбцами.
\end{definition}
\begin{theorem}
	Столбцы матрицы $A$, входящие в БМ, образуют ЛНЗ систему. Любой столбец матрицы $A$ линейно выражается через столбцы из БМ.
\end{theorem}
\subsection{Вычисление обратной матрицы с помощью присоединённой (союзной) матрицы.}
Нахождение обратной матрицы методом алгебраических дополнений:
\begin{enumerate}
	\item Вычисляем определитель (должен быть не равен 0).
	\item Составляем матрицу алгебраических дополнений. 
	\item Транспонируем полученную матрицу (в $\mathbb{C}$ находим комплексно сопряжённую (звёздочка в формуле ниже) и потом уже транспонируем).
	\item Делим результат на определитель исходной матрицы.
\end{enumerate}
\[A^{-1}=\frac{1}{\Delta A} A^{*^t}\]
\begin{evidence}
	Пусть $\Delta = \det A \ne 0$. Составим матрицу $B$ из алгебраических дополнений матрицы $A$, расположенных по столбцам, то есть матрицу вида:
	\[
	B=\frac{1}{\Delta}
	\begin{pmatrix}
	A_{11} & A_{21} & ... & A_{n1} \\
	A_{12} & A_{22} & ... & A_{n2} \\
	... & ... & ... & ... \\
	A_{1n} & A_{2n} & ... & A_{nn} \\
	\end{pmatrix} - \text{присоединённая матрица.}
	\]
	Составим произведение:
	\[
	B \cdot A = \frac{1}{\Delta}
	\begin{pmatrix}
	\sum\limits_{j=1}^nA_{j1}a_{j1} & \sum\limits_{j=1}^nA_{j1}a_{j2} & ... & \sum\limits_{j=1}^nA_{j1}a_{jn} \\
	\sum\limits_{j=1}^nA_{j2}a_{j1} & \sum\limits_{j=1}^nA_{j2}a_{j2} & ... & \sum\limits_{j=1}^nA_{j2}a_{jn} \\
	... & ... & ... & ... \\
	\sum\limits_{j=1}^nA_{jn}a_{j1} & \sum\limits_{j=1}^nA_{jn}a_{j2} & ... & \sum\limits_{j=1}^nA_{jn}a_{jn} \\
	\end{pmatrix}
	\]
	Тогда $B \cdot A = \frac{1}{\Delta} \cdot diag(\Delta, \Delta, ..., \Delta) = E$, поскольку все внедиагональные элементы данной матрицы равны нулю как разложения по «чужой» строке\footnote{Сумма произведений элементов $a_{ij}$ $i$-ой строки на алгебраические дополнения $A_{kj}$ элементов "чужой" $k$-ой строки при $j \ne k$ равна нулю: $\sum\limits_{j=1}^na_{ij}A_{kj}=a_{i1}A_{k1}+...+a_{in}A_{kn}=0$.}. Таким образом, матрица В является обратной по отношению к матрице $A$.
\end{evidence}
\subsection{Определитель произведения двух матриц.}
\noindent Аксиоматическое определение определителя:
\begin{enumerate}
	\item кососимметрическая (ассиметрическая) функция строк (столбцов) матрицы;
	\item полилинейная функция строк (столбцов) матрицы;
	\item $\det E = 1$.
\end{enumerate}
\noindent Ввводим функцию $f(x) = \frac{\det (XA)}{\det A}$. Докажем, что она обладает всеми свойствами определителя:
\begin{itemize}
	\item Полилинейность: $\dfrac{\det ((M_k(\alpha)X)A)}{\det A} = \dfrac{\det (M_k(\alpha)(XA))}{\det A} = \alpha \cdot f(x)$. Перегруппируем за счёт ассоциативности умножения, а выносим по свойству определителя.
	\item Ассиметричность аналогично: $f(E) = \dfrac{\det (E \cdot A)}{\det A} = 1$.
\end{itemize}
\subsection{Системы линейных алгебраических уравнений (СЛАУ), матричная завпись. Частное и общее решения.}
Система линейных алгебраических уравнений - система уравнений, каждое уравнение в которой является линейным — алгебраическим уравнением первой степени.
Частное решение СЛАУ – конкретный набор значений переменных $x_1, x_2, ..., x_n$, при подстановке которых в каждое уравнение системы, получается верное равенство.
\[
\begin{cases}
	a_{11}x_1 + a_{12}x_2 + ... + a_{1n}x_n = b_1 \\
	a_{21}x_1 + a_{22}x_2 + ... + a_{2n}x_n = b_2 \\
	... \\
	a_{m1}x_1 + a_{m2}x_2 + ... + a_{mn}x_n = b_m
\end{cases}
\]
\noindent Здесь $m$ - количество уравнений, а $n$ - количество переменных, $x_1, x_2, ..., x_n$ - неизвестные, которые надо определить, коэффициенты $a_{11}, a_{12}, ..., a_{mn}$ и  свободные члены $b_1, b_2, ..., b_m$ предполагаются известными.
Система линейных алгебраических уравнений может быть представлена в матричной форме как:
\[
\begin{pmatrix}
a_{11} & a_{12} & ... & a_{1n} \\
a_{21} & a_{22} & ... & a_{2n} \\
... & ... & ... & ... \\
a_{m1} & a_{m2} & ... & a_{mn}
\end{pmatrix}
\begin{pmatrix}
x_1 \\ x_2 \\ ... \\ x_n
\end{pmatrix}
=
\begin{pmatrix}
b_1 \\ b_2 \\ ... \\ b_m
\end{pmatrix}
\]
\noindent Или: $Ax=b$. Здесь $A$ - матрица системы, $x$ -  столбец неизвестных, $b$ - столбец свободных членов. Если к матрице $A$ приписать справа столбец свободных членов, то получившаяся матрица называется расширенной.
Общее решение неоднородной СЛАУ равно сумме частного решения неоднородной СЛАУ и общего решения соответствующей однородной СЛАУ. Частное решение можно получить из общего подстановкой определённого значения вместо свободных членов.
\subsection{Формула Крамера.}
Пусть матричное уравнение $Ax=b$ описывает систему $n$ линейных уравнений с $n$ неизвестными. Если $\det A \ne 0$, то система является совестной и имеет единственное решение: $x_i = \frac{\Delta_i}{\Delta}, i=1,2,...,n$. Где $\Delta = \det A, \Delta_i$ - определитель, полученный из определителя $\Delta$ заменой $i$-ого столбца столбцом свободных членов матрицы $b$.
\[
\Delta_i=
\begin{vmatrix}
a_{11} & a_{1, i-1} & b_1 & a_{1, i+1} & ... & a_{1n} \\
... & ... & ... & ... & ... & ... \\
a_{n1} & a_{n, i-1} & b_n & a_{n, i+1} & ... & a_{nn}
\end{vmatrix}
\]
\begin{evidence}
	Т.к. $\det A \ne 0 \rightarrow \exists ! A^{-1}$. Умножая обе части уравнения на неё, получаем решение: $X=A^{-1}b$. Составляем присоединённую матрицу
	\[
	\tilde{A} = 
	\begin{pmatrix}
	A_{11} & A_{21} & ... & A_{n1} \\
	A_{12} & A_{22} & ... & A_{n2} \\
	... & ... & ... & ... \\
	A_{1n} & A_{2n} & ... & A_{nn}
	\end{pmatrix}
	\] и $X = \frac{1}{\Delta} \tilde{A}b$. Т.е. $x_i = (A^{-1}b)_i = \frac{1}{\Delta}(A_{1i}, A_{2i}, ..., A_{ni})$.
	
	\noindent $\begin{pmatrix}b_1 \\ b_2 \\ ... \\ b_n\end{pmatrix} = \frac{1}{\Delta} \sum\limits_{k=1}^nA_{ki}b_k$. Сумма в правой части этого равенства представляет собой разложение определителя $\Delta_i$ по элементам $i$-го столбца и, следовательно, $x_i = \dfrac{\Delta_i}{\Delta}$.
\end{evidence}
\paragraph{Решение системы двух уравнений методом Крамера:}
\[
\begin{cases}
	a_1x + b_1y = s_1 \\
	a_2x + b_2y = s_2
\end{cases}
\]
\begin{enumerate}
	\item $\Delta = \begin{vmatrix}a_1 & b_1 \\ a_2 & b_2\end{vmatrix}$
	\item Если $\Delta \ne 0$, то $\Delta_x = \begin{vmatrix}s_1 & b_1 \\ s_2 & b_2\end{vmatrix}$ и $\Delta_y = \begin{vmatrix}a_1 & s_1 \\ a_2 & s_2\end{vmatrix}$
	\item $x = \frac{\Delta_x}{\Delta}, y = \frac{\Delta_y}{\Delta}$
\end{enumerate}
\subsection{Метод Гаусса решения систем линейных уравнений. Теорема Кронкера-Капелли.}
Прямой ход Гаусса — процесс последовательного исключения неизвестных.
Обратный ход Гаусса — процесс последовательного нахождения неизвестных от последнего уравнения к первому.
\noindent \textbf{Алгоритм:}
\begin{enumerate}
	\item Определяем, что определитель матрицы не равен 0.
	\item Записываем расширенную матрицу данной системы линейных уравнений.
	\item На первом этапе (прямой ход) система приводится ступенчатой или треугольной форме. Вычтем из второго уравнения системы первое, умноженное на такое число, чтобы обнулился коэффициент при $x_1$. Затем таким же образом вычтем первое уравнение из третьего, четвертого и т.д. Тогда исключаются все коэффициенты первого столбца, лежащие ниже главной диагонали. Затем при помощи второго уравнения исключим из третьего, четвертого и т.д. уравнений коэффициенты второго столбца. Последовательно продолжая этот процесс, исключим из матрицы все коэффициенты, лежащие ниже главной даигонали.
	\item Проверяем систему на совместимость.
	\item Вычисляем кол-во свободных переменных (членов): $n-r$ (количество неизвестных (столбцы) - ранг полученной матрицы).
	\item На втором этапе (обратный ход) выражаем все получившиеся базисные переменные через небазисные и построим фундаментальную систему решений. Если все переменные являются базисными, то получим единственное решение системы линейных уравнений. Эта процедура начинается с последнего уравнения, из которого выражают соответствующую базисную переменную (а она там всего одна) и подставляют в предыдущие уравнения, и так далее, поднимаясь по «ступенькам» наверх. Каждой строчке соответствует ровно одна базисная переменная, поэтому на каждом шаге, кроме последнего (самого верхнего), ситуация в точности повторяет случай последней строки.
\end{enumerate}
\noindent Общее решение совместной системы линейных уравнений - это новая система, равносильная исходной, в которой все разрешенные переменные выражены через свободные.
\paragraph{Теорема Кронекера-Капелли (условие совместности системы)}
Рассмотрим систему $m$ линейных алгебраических уравнений (СЛАУ) с $n$ неизвестными:
\[
\begin{cases}
	a_{11}x_1 + a_{12}x_2 + ... + a_{1n}x_n = b_1 \\
	a_{21}x_1 + a_{22}x_2 + ... + a_{2n}x_n = b_2 \\
	... \\
	a_{m1}x_1 + a_{m2}x_2 + ... + a_{mn}x_n = b_m
\end{cases}
\]
\noindent Выпишем основную матрицу этой системы $A$ и расширенную матрицу $\bar A$:
\[
A=
\begin{pmatrix}
	a_{11} & a_{12} & ... & a_{1n} \\
	a_{21} & a_{22} & ... & a_{2n} \\
	... & ... & ... & ... \\
	a_{m1} & a_{m2} & ... & a_{mn} \\
\end{pmatrix},
\bar A =
\begin{pmatrix}
	a_{11} & a_{12} & ... & a_{1n} & b_1 \\
	a_{21} & a_{22} & ... & a_{2n} & b_2 \\
	... & ... & ... & ... & ... \\
	a_{m1} & a_{m2} & ... & a_{mn} & b_m \\
\end{pmatrix}
\]
\noindent СЛАУ совместна тогда и только тогда, когда ранг её основной матрицы $A$ равен рангу её расширенной матрицы $\bar A: r(A) = r(\bar A)$.
\noindent Причём система имеет единственное решение, если ранг равен числу неизвестных $r(A) = r(\bar A) = n$ и бесконечное множество решений, если ранг меньше числа неизвестных $r(A) = r(\bar A) < n$.
\begin{evidence}
	Пояснение: система уравнений $Ax=b$ разрешима тогда и только тогда, когда $rang A = rang (A, b)$, где $(A, b)$ расширенная матрица, полученная из матрицы $A$ приписыванием столбца $b$.
	
	\textit{Необходимость:} пусть система совместна. Тогда существуют числа $\lambda_1,...,\lambda_n \in \mathbb{R}$ такие, что $b = \lambda_1a_1 + ... + \lambda_na_n$. Следовательно, столбец $b$ является линейной комбинацией столбцов $a_1,...,a_n$ матрицы $A$. Из того, что ранг матрицы не изменится, если из системы её строк (столбцов) вычеркнуть или приписать строку (столбец), которая является линейной комбинацией других строк (столбцов) следует, что $rang A = rang B$.
	
	\textit{Достаточность:} пусть $rang A = rang B = r$. Возьмём в матрице $A$ какой-нибудь базисный минор. Так как $rang B = r$, то он же будет базисным минором и матрицы $B$. Тогда, согласно теореме о базисном миноре, последний столбец матрицы $B$ будет линейной комбинацией базисных столбцов, то есть столбцов матрицы $A$. Следовательно, столбец свободных членов системы является линейной комбинацией столбцов матрицы $A$.
\end{evidence}
\subsection{Однородные системы линейных уравнений, свойства решений.}
Если все свободные члены системы линейных уравнений равны 0, то система называется однородной, в противном случае — неоднородной. Если в системе все свободные члены заменить нулями, то мы получим
однородную систему:
\[
\begin{cases}
a_{11}x_1 + a_{12}x_2 + ... + a_{1n}x_n = 0 \\
a_{21}x_1 + a_{22}x_2 + ... + a_{2n}x_n = 0 \\
... \\
a_{m1}x_1 + a_{m2}x_2 + ... + a_{mn}x_n = 0 \\
\end{cases}
\]
которую будем называть однородной системой, соответствующей исходной системе.
\begin{remark}
	любая однородная система линейных уравнений совместна.
\end{remark}
\begin{definition}
	Пусть $(x_1, x_2, ..., x_n) \text{ и } (y_1, y_2, ..., y_n)$ - два упорядоченных набора чисел, а $t$ — некоторое число. Тогда упорядоченный набор чисел $x_1 + y_1, x_2 + y_2, ..., x_n + y_n$ называется суммой наборов $x$ и $y$, а упорядоченный набор $(tx_1, tx_2, ..., tx_n)$ - произведением набора $x$ на число $t$.
\end{definition}
\begin{theorem}
	Сумма любых двух частных решений однородной системы линейных уравнений является решением этой системы. Произведение любого частного решения однородной системы линейных уравнений на любое число является решением этой системы.
	\begin{evidence}
		Пусть $(y_1, y_2, ..., y_n)$ и $(z_1, z_2, ..., z_n)$ - решения системы. Подставив числа $z_1 + y_1, z_2 + y_2, ..., z_n + y_n$ в $i$-ое уравнение этой системы (где $1 \le i \le m$) получим
		\begin{multline}
			a_{i1}(z_1 + y_1) + a_{i2}(z_2 + y_2) + ... + a_{in}(z_n + y_n) = \\
			= (a_{i1}z_1 + a_{i2}z_2 + ... + a_{in}z_n) + (a_{i1}y_1 + a_{i2}y_2 + ... + a_{in}y_n) = \\ 
			= 0 + 0 = 0
		\end{multline}
		Подставив в то же уравнение числа $ty_1, ty_2, ..., ty_n$ получим в результате $t \cdot 0 = 0$. Значит, в обоих случаях они являются решениями системы.
	\end{evidence}
\end{theorem}
\begin{corollary}
	Произвольная однородная система линейных уравнений либо имеет ровно одно решение, либо имеет бесконечно много решений.
\end{corollary}
\begin{theorem}
	Пусть система линейных уравнений совместна. Выберем произвольным образом и зафиксируем некоторое ее частное решение $x_1, x_2, ..., x_n$.
	\[
	\begin{cases}
	a_{11}x_1 + a_{12}x_2 + ... + a_{1n}x_n = 0 \\
	a_{21}x_1 + a_{22}x_2 + ... + a_{2n}x_n = 0 \\
	... \\
	a_{m1}x_1 + a_{m2}x_2 + ... + a_{mn}x_n = 0 \\
	\end{cases}
	\]
	\begin{enumerate}
		\item Если $(y_1, y_2, ..., y_n)$ - частное решение системы, в которой все свободные члены заменены 0 (однородная система, соответствующая исходной), то сумма наборов чисел $(x_1, x_2, ..., x_n)$ и $(y_1, y_2, ..., y_n)$ является частным решением системы.
		\item Обратно, каждое частное решение системы является суммой решения $(x_1, x_2, ..., x_n)$ этой системы и некоторого частного решения системы, в которой все свободные члены заменены 0 (однородная система, соответствующая исходной).
	\end{enumerate}
	\begin{evidence} \leavevmode
		\begin{enumerate}
			\item Подставим числа $x_1 + y_1, x_2 + y_2, ..., x_n + y_n$ в произвольное $i$-ое уравнение системы (где $1 \le i \le m$). Получим
			\begin{multline}
			a_{i1}(x_1 +\ y_1)+a_{i2}(x_2 +\ y_2)+\ldots+a_{in}(x_n +\ y_n) = \\ =\left(a_{i1}x_1+a_{i2}x_2+\ldots+a_{in}x_n\right)+\left(a_{i1}y_1+a_{i2}y_2+\ldots+a_{in}y_n\right)= \\ =b_i+0=b_i
			\end{multline} Значит, исходный набор – решение системы.
			\item Пусть $\left(z_1,z_2,\dots,z_n\right)$ - частное решение системы. Положим $y_1=z_1-x_1,\ y_2=z_2-x_2,\dots,\ y_n=z_n-x_n$. Подставим числа $y_1, y_2, ..., y_n$ в произвольное $i$-ое уравнение системы (где $1 \le i \le m$). Получим
			\begin{multline}
			a_{i1}y_1+a_{i2}y_2+\dots+a_{in}y_n = \\ =a_{i1}\left(z_1-x_1\right)+a_{i2}\left(z_2-x_2\right)+\dots+a_{in}\left(z_n-x_n\right)= \\ =b_i-b_i=0
			\end{multline} Значит набор $y$ – частное решение системы. В итоге: однородной система, соответствующей исходной.
		\end{enumerate}	
	\end{evidence}
\end{theorem}
\begin{corollary}
	Произвольная система линейных уравнений либо не имеет решений, либо имеет ровно одно решение, либо имеет бесконечно много решений.
\end{corollary}
\begin{remark} \leavevmode
	\begin{itemize}
		\item общее решение есть у любой системы. В частности, у несовместной системы общим решением является пустое множество.
		\item общее решение (совместной) системы линейных уравнений равно сумме ее частного решения и общего решения соответствующей однородной системы.
	\end{itemize}
\end{remark}
\subsection{Фундаментальная система решений однородных линейных уравнений.}
Фундаментальная система решений – это множество линейно независимых векторов $\bar{a_1}, \bar{a_2}, ..., \bar{a_n}$, каждый из которых является решением однородной системы, кроме того, решением также является линейная комбинация данных векторов $c_1\bar{a_1}+c_2\bar{a_2}+\ldots+c_n\bar{a_n}$, где $c_1,c_2,\dots,c_n$ - произвольные действительные числа. 
Количество векторов $n$ фундаментальной системы рассчитывается по формуле: $n=$ Количество неизвестных в системе – Ранг матрицы системы ИЛИ количество векторов $n$ фундаментальной системы равно количеству свободных неизвестных.
\begin{theorem}
	Если имеется система однородных линейных уравнений $\Delta X = 0 \rightarrow A \in M_{m \times n}, rang A = r \rightarrow \exists n - r$ линейно независимых решений, таких, что $X_0 = c_1X_1 + c_2X_2 + ... + c_{n-r}X_{n-r}$ - фундаментальная система решений.
\end{theorem}
\paragraph{Терминология решений СЛАУ:}
\begin{itemize}
	\item СЛАУ называется совместной, если она имеет, хотя бы одно решение. В противном случае система называется несовместной.
	\item Система называется определённой, если она совместна и имеет единственное решение. В противном случае (т.е. если система совместна и имеет более одного решения) система называется неопределённой.
	\item Система называется однородной, если все правые части уравнений, входящих в нее, равны нулю одновременно.
	\item Для неоднородных системы линейных алгебраических уравнений общее решение представляется в виде $X_{\text{орнослау}} = X_{\text{орослау}} + X_{\text{чрнслау}}$ (орослау - общее решение однородной СЛАУ, орнослау - общее решение неоднородной СЛАУ, чрнслау - частное решение неоднородной СЛАУ). Где $X_{\text{орослау}} = c_1X_1 + c_2X_2 + ... + c_{n-r}X_{n-r}$, а $X_{\text{чрнслау}}$ - частное решение исходной неоднородной СЛАУ, которое мы получаем, придав свободным неизвестным значения $0,0,…,0$ и вычислив значения основных неизвестных.
	\item Фундаментальной системой решений однородной системы из p линейных алгебраических уравнений с $n$ неизвестными переменными называют совокупность $(n – r)$ линейно независимых решений этой системы, где $r$ – порядок базисного минора основной матрицы системы. $X_0 = c_1X_1 + c_2X_2 + ... + c_{n-r}X_{n-r}$.
\end{itemize}
\subsection{Структура общего решения СЛАУ.}
Для записи общего решения после нахождения всех несвободных переменных (выразили через свободные и константы), надо:
\begin{enumerate}
	\item записать матрицу-столбец из чисел в каждом решении;
	\item последовательно записывать вынесенную из каждого решения свободную переменную с коэффициентом, равным матрице-столбцу со значениями – коэффициентами этой переменной в каждом решении.
\end{enumerate}
\begin{exmp}
	Найти фундаментальную систему решений и общее решение однородной системы
	\[
	\begin{cases}
	x_1 + x_2 + 2x_3 + x_4 = 0 \\
	2x_1 + 3x_2 + x_4 = 0 \\
	3x_1 + 4x_2 + 2x_3 + 2x_4 = 0
	\end{cases}
	\]
	\begin{nonum}
		Составим расширенную матрицу системы:
		\[
		A|0 = 
		\begin{pmatrix}
		1 & 1 & 2 & 1 & | & 0 \\
		2 & 3 & 0 & 1 & | & 0 \\
		3 & 4 & 2 & 2 & | & 0
		\end{pmatrix}
		\]
		Используя элементарные преобразования над строками, приведём матрицу к ступенчатому, а затем и к упрощённому виду:
		\[
		A'|0 = 
		\begin{pmatrix}
		1 & 0 & 6 & 2 & | & 0 \\
		0 & 1 & -4 & -1 & | & 0 \\
		0 & 0 & 0 & 0 & | & 0
		\end{pmatrix}
		\]
		Переменные $x_1, x_2$ - базисные, $x_3, x_4$ - свободные. Записываем формулу общего однородной системы:
		\[
		\begin{cases}
		x_1 = -6x_3 - 2x_4 \\
		x_2 = 4x_3 + x_4
		\end{cases}
		\]
		Далее находим фундаментальну систему решений. Т.к. $n=4$ и $r = rang A = 2$, надо подобрать $n - r = 2$ линейно независимых решения. Пожставляя в систему стандартные наборы значений свободных переменных:
		\begin{enumerate}
			\item если $x_3 = 1, x_4 = 0$, то $x_1 = -6, x_2 = 4$;
			\item если $x_3 = 0, x_4 = 1$, то $x_1 = -2, x_2 = 1$.
		\end{enumerate}
		В результате получили фундаментальную систему решений:
		\[
		\phi_1 = \begin{pmatrix}-6 \\ 4 \\ 1 \\ 0\end{pmatrix}, \phi_2 = \begin{pmatrix}-2 \\ 1 \\ 0 \\ 1\end{pmatrix}
		\]
		Записываем общее решение системы:
		\[
		x = C_1 \cdot \begin{pmatrix}-6 \\ 4 \\ 1 \\ 0\end{pmatrix} + C_2 \cdot \begin{pmatrix}-2 \\ 1 \\ 0 \\ 1\end{pmatrix}
		\]
		Заметим, что фундаментальную систему решений можно получить, взяв иные наборы значений переменных.
	\end{nonum}
\end{exmp}
\newpage
\tableofcontents
\end{document} % конец документа