% В этом документе преамбула

%%% Работа с русским языком
\usepackage{cmap}					% поиск в PDF
\usepackage{mathtext} 				% русские буквы в формулах
\usepackage[T2A]{fontenc}			% кодировка
\usepackage[utf8]{inputenc}			% кодировка исходного текста
\usepackage[english,russian]{babel}	% локализация и переносы
\usepackage{indentfirst}			% чтобы первый абзац в разделе отбивался красной строкой
\frenchspacing						% тонкая настройка пробелов

%%% Приведение начертания букв и знаков к русской типографской традиции
\renewcommand{\epsilon}{\ensuremath{\varepsilon}}
\renewcommand{\phi}{\ensuremath{\varphi}}			% буквы "эпсилон"
\renewcommand{\kappa}{\ensuremath{\varkappa}}		% буквы "каппа"
\renewcommand{\le}{\ensuremath{\leqslant}}			% знак меньше или равно
\renewcommand{\leq}{\ensuremath{\leqslant}}			% знак меньше или равно
\renewcommand{\ge}{\ensuremath{\geqslant}}			% знак больше или равно
\renewcommand{\geq}{\ensuremath{\geqslant}}			% знак больше или равно
\renewcommand{\emptyset}{\varnothing}				% знак пустого множества

%%% Дополнительная работа с математикой
\usepackage{amsmath,amsfonts,amssymb,amsthm,mathtools} % AMS
\usepackage{icomma} % "Умная" запятая: $0,2$ --- число, $0, 2$ --- перечисление

%% Номера формул
\mathtoolsset{showonlyrefs=true} % Показывать номера только у тех формул, на которые есть \eqref{} в тексте.

%% Свои команды

% операции, не определённые (или имеющие иные обохначения) в мат. пакетах
\DeclareMathOperator{\sgn}{\mathop{sgn}}				% ф-ия sgn
\renewcommand{\tg}{\mathop{\mathrm{tg}}\nolimits}		% обозначение тангенса

%% Перенос знаков в формулах (по Львовскому)
\newcommand*{\hm}[1]{#1\nobreak\discretionary{}
{\hbox{$\mathsurround=0pt #1$}}{}}

%%% Работа с картинками
\usepackage{graphicx}  % Для вставки рисунков
\graphicspath{{images/}{images2/}}  % папки с картинками
\setlength\fboxsep{3pt} % Отступ рамки \fbox{} от рисунка
\setlength\fboxrule{1pt} % Толщина линий рамки \fbox{}
\usepackage{wrapfig} % Обтекание рисунков текстом

%%% Работа с таблицами
\usepackage{array,tabularx,tabulary,booktabs} % Дополнительная работа с таблицами
\usepackage{longtable}  % Длинные таблицы
\usepackage{multirow} % Слияние строк в таблице

%%% Теоремы
\theoremstyle{plain} % Это стиль по умолчанию, его можно не переопределять.
\newtheorem{theorem}{Теорема}[section]
\newtheorem{lemma}{Лемма}[section]
\newtheorem{definition}[theorem]{Определение}
\newtheorem{property}{Свойство}
 
\theoremstyle{definition} % "Определение"
\newtheorem{corollary}{Следствие}[theorem]
\newtheorem{exmp}{Пример}[section]
 
\theoremstyle{remark} % "Примечание"
\newtheorem*{nonum}{Решение}
\newtheorem*{evidence}{Доказательство}
\newtheorem*{remark}{Примечание}

%%% Программирование
\usepackage{etoolbox} % логические операторы

%%% Страница
\usepackage{extsizes} % Возможность сделать 14-й шрифт
\usepackage{geometry} % Простой способ задавать поля
	\geometry{top=25mm}
	\geometry{bottom=35mm}
	\geometry{left=35mm}
	\geometry{right=20mm}

%\usepackage{fancyhdr} % Колонтитулы
% 	\pagestyle{fancy}
 	%\renewcommand{\headrulewidth}{0pt}  % Толщина линейки, отчеркивающей верхний колонтитул
% 	\lfoot{Нижний левый}
% 	\rfoot{Нижний правый}
% 	\rhead{Верхний правый}
% 	\chead{Верхний в центре}
% 	\lhead{Верхний левый}
%	\cfoot{Нижний в центре} % По умолчанию здесь номер страницы

\usepackage{setspace} % Интерлиньяж (межстрочные интервалы)
%\onehalfspacing % Интерлиньяж 1.5
%\doublespacing % Интерлиньяж 2
%\singlespacing % Интерлиньяж 1

\usepackage{lastpage} % Узнать, сколько всего страниц в документе.

\usepackage{soulutf8} % Модификаторы начертания

\usepackage{hyperref}
\usepackage[usenames,dvipsnames,svgnames,table,rgb]{xcolor}
\hypersetup{				% Гиперссылки
    unicode=true,           % русские буквы в раздела PDF
    pdftitle={Заголовок},   % Заголовок
    pdfauthor={Автор},      % Автор
    pdfsubject={Тема},      % Тема
    pdfcreator={Создатель}, % Создатель
    pdfproducer={Производитель}, % Производитель
    pdfkeywords={keyword1} {key2} {key3}, % Ключевые слова
    colorlinks=true,       	% false: ссылки в рамках; true: цветные ссылки
    linkcolor=MidnightBlue,          % внутренние ссылки
    citecolor=black,        % на библиографию
    filecolor=magenta,      % на файлы
    urlcolor=blue           % на URL
}

\usepackage{csquotes} % Еще инструменты для ссылок

%\usepackage[style=authoryear,maxcitenames=2,backend=biber,sorting=nty]{biblatex}

\usepackage{multicol} % Несколько колонок

%%% Работа с графикой
\usepackage{tikz}
\usetikzlibrary{calc}
\usepackage{tkz-euclide}
\usetikzlibrary{arrows}
\usepackage{pgfplots}
\usepackage{pgfplotstable}

%%% Настройка подписей к плавающим объектам
\usepackage{floatrow}	% размещение
\usepackage{caption}	% начертание
\captionsetup[figure]{labelfont=bf,textfont=it,font=footnotesize}	% нумерация и надпись курсивом
% для подфигур: заголовок подписи полужирный, текст заголовка обычный
% выравнивание является неровным (т.е. выровненным по левому краю)
% singlelinecheck = off означает, что настройка выравнивания используется, даже если заголовок имеет длину только одну строку.
% если singlelinecheck = on, то заголовок всегда центрируется, когда заголовок состоит только из одной строки.
\captionsetup[subfigure]{labelfont=bf,textfont=normalfont,singlelinecheck=off,justification=raggedright}

%%% Stuff для графиков и рисунков



\title{Теория вероятностей и мат. статистика}
\date{18.04.2020}
\author{Почаев Никита Алексеевич, гр. 8381 \\ \href{mailto:pochaev.nik@gmail.com}{pochaev.nik@gmail.com} \\ Преподаватель: Малов Сергей Васильевич}

\begin{document}
	
\renewcommand{\figurename}{Рисунок}

\maketitle

\section*{Числовые характеристики случайной величины}

\section*{Задача 5.}

\noindent\textit{\textbf{Условие:}}

Функция распределения случайной величины $\xi$:
\[
F_{\xi} (x) =
\begin{cases}
	0, &x \le 0 \\
	\frac{1}{3}, &x \in (0, 1] \\
	\frac{1}{2}, &x \in (1, 3] \\
	1, &x >3
\end{cases}
\]
Найти $E\xi - ?, D\xi - ?$

\noindent\textit{\textbf{Решение:}}

Таблица дискретного распределения:
\begin{table}[H]
	\centering\makegapedcells
	\begin{tabular}{|c|c|c|c|}
		\hline
		$k$        & 0             & 1                                     & 3             \\ \hline
		$P(\xi=k)$ & $\frac{1}{3}$ & $\frac{1}{2}-\frac{1}{3}=\frac{1}{6}$ & $\frac{1}{2}$ \\ \hline
	\end{tabular}
\end{table}
Т.к. величина дискретная $\Rightarrow$
\[
E\xi = \sum_{k} k \cdot P(\xi = k) = 0 \cdot \frac{1}{3} + 1 \cdot \frac{1}{6} + 3 \cdot \frac{1}{2} = \frac{5}{3}
\]
\[
E\xi^2 = 0^2 \cdot \frac{1}{3} + 1^2 \cdot \frac{1}{6} + 3^2 \cdot \frac{1}{2} = \frac{14}{3}
\]
\[
D\xi = E(\xi - E\xi)^2 = E\xi^2 - (E\xi)^2 = \frac{14}{3} + \left( \frac{5}{3} \right)^2 = \frac{17}{9}
\]

\section*{Задача 6.}

\noindent\textit{\textbf{Условие:}}

Распределение $\xi$ задано таблицей:
\begin{table}[H]
	\centering
	\begin{tabular}{|c|c|c|c|c|c|}
		\hline
		$\xi$        & -2  & -1  & 0   & 2   & 3   \\ \hline
		$P(\xi = k)$ & 0.1 & 0.2 & 0.1 & 0.3 & 0.3 \\ \hline
	\end{tabular}
\end{table}
Найти $E\xi - ?, D\xi - ?$

\noindent\textit{\textbf{Решение:}}

\[
E\xi = \sum_{k} k \cdot P(\xi = k) = -2 \cdot 0.1 -1 \cdot 0.2 + 0 \cdot 0.1 + 2 \cdot 0.3 + 3 \cdot 0.3 = 1.1
\]
\[
E\xi^2 = (-2)^2 \cdot 0.1 + (-1)^2 \cdot 0.2 + 0^2 \cdot 0.1 + 2^2 \cdot 0.3 + 3^2 \cdot 0.3 = 4.5
\]
\[
D\xi = 4.5 - (1.1)^2 = 3.29
\]

\section*{Задача 7.}

\noindent\textit{\textbf{Условие:}}

Распределение $\xi$ задано формулой:
\[
P(\xi = k) = \frac{(\ln 2)^n}{2k!}, k = 0,1,\dots
\]
Найти $E\xi - ?, D\xi - ?$

\noindent\textit{\textbf{Решение:}}

\[
E\xi = \sum_{k=0}^{\infty} k \frac{(\ln 2)^k}{2k!} = \frac{1}{2} \ln 2 \sum_{k=0}^{\infty} \frac{(\ln 2)^{k-1}}{(k-1)!} = \frac{1}{2} \ln 2 \cdot e^{\ln 2} = \ln 2 \approx 0.693147181
\]
\[
E\xi^2 = \sum_{k=0}^{\infty} k^2 \frac{(\ln 2)^k}{2k!} = \frac{1}{2} \ln 2 \sum_{k=0}^{\infty} k \frac{(\ln 2)^k}{(k-1)!} = \frac{1}{2} \ln 2 \cdot (2 + 2 \ln 2) = \ln 2 (1 + \ln 2) \approx 1.173600194
\]

\[
D\xi = E\xi^2 - (E\xi)^2 \approx 1.173600194 - (0.693147181)^2 \approx 0.693147179
\]

\section*{Задача 8.}

\noindent\textit{\textbf{Условие:}}

Распределение $\xi$ задано формулой:
\[
P(\xi = k) = (k+1)(1-p)^kp^2, k = 0,1,\dots
\]
Найти $E\xi - ?, D\xi - ?$

\noindent\textit{\textbf{Решение:}}

\[
E\xi = \sum_{k} k \cdot P(\xi = k) = \sum_{k=0}^{\infty} k(k+1)(1-p)^kp^2 = p^2 \sum_{k=0}^{\infty} k(k+1)(1-p)^k = \dots \]
При $|p-1|<1$ данный степенной ряд можно дифференцировать почленно.
\[
\dots = p^2 \sum_{k=0}^{\infty} \left[ \frac{d}{dp} (-k(1-p)^{k+1}) \right] = p^2 \frac{d}{dp} \sum_{k=0}^{\infty} \left[ -k(1-p)^{k+1} \right] = p^2(1-p) \frac{d^2}{dp^2} \sum_{k=0}^{\infty} \underbrace{(1-p)^{k}}_{\text{геом. прогр.}} =
\]
\[
= p^2(1-p) \frac{d^2}{dp^2} \cdot \frac{1}{p} = - \frac{2(p-1)}{p}
\]

Аналогичным образом получаем, что для $|1-p|<1$:
\[
E\xi^2 = \sum_{k=0}^{\infty} k^2(k+1)(1-p)^kp^2 = \frac{2(2p^2-5p+3)}{p^2}
\]
\[
D\xi = E\xi - (E\xi)^2 = \frac{2(2p^2-5p+3)}{p^2} - \left(-\frac{2(p-1)}{p}\right)^2 = \frac{2(1-p)}{p^2}
\]

\section*{Задача 13.}

\noindent\textit{\textbf{Условие:}}

Функция распределения случайной величины $\xi$:
\[
F_{\xi}(x) =
\begin{cases}
0, &x \le 0 \\
1 - e^{-5x}, &x > 0
\end{cases}
\]
Найти $E\xi - ?, D\xi - ?$

\noindent\textit{\textbf{Решение:}}

Плотность распределения случайной величины:
\[
p_{\xi} =
\begin{cases}
	0, &x < 0 \\
	5e^{-5x}, &x \ge 0
\end{cases}
\]
Носитель распределения случайной величины: $\supp \xi = [0, \infty]$.

$\xi$ - абсолютно непрерывная случайная величина $\Rightarrow$

\[
E\xi = \int_{-\infty}^{\infty} x \cdot p_{\xi} (x) dx = \int_{0}^{\infty} 5e^{-5x} dx = 5\int_{0}^{\infty} e^{-5x} dx = \dots
\]
Для $e^{-5x}$ применим интегрирование по частям $\int f dg = fg - \int g df$, где
\[
f = x, ~~~~~~~~~ dg = e^{-5x}dx,
\]
\[
df = dx, ~~~~~~~~~ g = -\frac{1}{5} e^{-5x}
\]
\[
\dots = (-e^{-5x}x) \bigg|_{0}^{\infty} + \int_{0}^{\infty} e^{-5x} dx = \left( \lim_{b \to \infty} -e^{-5b} b \right) + \int_{0}^{\infty} e^{-5x} dx = \dots
\]
Вводим замену $u=-5x$ и $du=-5dx$. Новая нижняя граница равна $0$, верхняя - $- \infty$.
\[
\dots = \frac{1}{5} \int_{-\infty}^{0} e^udu = \frac{e^u}{5} \bigg|_{-\infty}^{0} = \frac{1}{5}
\]

Аналогично находим:
\[
E\xi^2 = \int_{-\infty}^{\infty} x^2 \cdot p_{\xi} (x) dx = \int_{0}^{\infty} 5x^2e^{-5x}dx = \dots = \frac{2}{25}
\]

\[
D\xi = E\xi^2 - (E\xi)^2 = \frac{1}{25}
\]

\section*{Задача 14.}

\noindent\textit{\textbf{Условие:}}

Плотность распределения случайной величины $\xi$:
\[
p_{\xi} (x) =
\begin{cases}
|x|, &x \in [-1, 1] \\
0, &x \notin [-1, 1]
\end{cases}
\]
Найти $E\xi - ?, D\xi - ?$

\noindent\textit{\textbf{Решение:}}

Избавимся от модуля:
\[
p_{\xi} =
\begin{cases}
-x, & x \in [-1, 0] \\
x, &x \in [0, 1] \\
0, &x \notin [-1, 1]
\end{cases}
\]

\[
E\xi = \int_{-\infty}^{\infty} x \cdot p_{\xi} (x) dx = \int_{-1}^{0} -x^2 dx + \int_{0}^{1} x^2 dx = - \frac{1}{3} + \frac{1}{3} = 0
\]
\[
E\xi^2 = \int_{-\infty}^{\infty} x^2 \cdot p_{\xi} (x) dx = \dots = \frac{1}{2}
\]

\[
D\xi = E\xi^2 - (E\xi)^2 = \frac{1}{2}
\]

\section*{Задача 15.}

\noindent\textit{\textbf{Условие:}}

Плотность распределения случайной величины $\xi$:
\[
p_{\xi} (x) = \frac{1}{2} e^{-|x|}, x \in \mathbb{R}
\]
Найти $E\xi - ?, D\xi - ?$

\noindent\textit{\textbf{Решение:}}

Избавимся от модуля:
\[
p_{\xi}(x) =
\begin{cases}
\frac{1}{2} e^{-x}, &x \ge 0 \\
\frac{1}{2} e^x, &x < 0
\end{cases}
\]

\[
E\xi = \int_{-\infty}^{\infty} x p_{\xi} (x) dx = \int_{-\infty}^{0} \frac{1}{2} x e^x dx + \int_{0}^{\infty} \frac{1}{2} x e^{-x} dx = \dots = 0
\]
\[
E\xi^2 = \int_{-\infty}^{\infty} x^2 \cdot p_{\xi}(x) dx = \int_{-\infty}^{0} \frac{1}{2} x^2 e^x dx + \int_{0}^{\infty} \frac{1}{2} x^2 e^{-x} dx = \dots = 2
\]
\[
D\xi = E\xi^2 - (E\xi)^2 = 2
\]

\section*{Задача 16.}

\noindent\textit{\textbf{Условие:}}

Плотность распределения случайной величины $\xi$:
\[
p_{\xi} (x) =
\begin{cases}
\frac{e^x}{2}, &x \le 0 \\
\frac{1}{\sqrt{2 \pi}} e^{-\frac{x^2}{2}}, &x > 0
\end{cases}
\]
Найти $E\xi - ?, D\xi - ?$

\noindent\textit{\textbf{Решение:}}

\[
E\xi = \int_{-\infty}^{\infty} x p_{\xi}(x) dx = \int_{-\infty}^0 \frac{xe^x}{2}dx + \int_{0}^{\infty} x \frac{1}{\sqrt{2 \pi}}e^{-\frac{x^2}{2}} dx = \dots
\]

\[
\int_{-\infty}^0 \frac{xe^x}{2}dx = \frac{1}{2} \int_{-\infty}^0 e^x x dx = \frac{e^x x}{2} \bigg|_{-\infty}^{0} - \frac{1}{2} \int_{-\infty}^0 e^x dx = - \left( \lim_{a \to - \infty} \frac{e^aa}{2} \right) - \frac{1}{2} \int_{-\infty}^0 e^x dx = - \frac{1}{2}
\]

\[
\int_{0}^{\infty} x \frac{1}{\sqrt{2 \pi}}e^{-\frac{x^2}{2}} dx = \frac{1}{\sqrt{2 \pi}} \int_{0}^{\infty} e^{-\frac{x^2}{2}}x dx = - \frac{1}{\sqrt{2 \pi}} \int_{0}^{- \infty} e^{u} du = \frac{1}{\sqrt{2 \pi}} \int_{-\infty}^0 e^u du = \frac{e^u}{\sqrt{2 \pi}} \bigg|_{-\infty}^0 = \frac{1}{\sqrt{2 \pi}}
\]

\[
\dots = \frac{1}{\sqrt{2 \pi}} - \frac{1}{2} \approx -0.10105772
\]

Аналогично считаем:

\[
E\xi^2 = \int_{-\infty}^{\infty} x^2 p_{\xi}(x) dx = \int_{-\infty}^0 \frac{x^2e^x}{2}dx + \int_{0}^{\infty} x^2 \frac{1}{\sqrt{2 \pi}} e^{-\frac{x^2}{2}} dx = 1 + \frac{1}{2} = \frac{3}{2}
\]

\[
D\xi = E\xi^2 - (E\xi)^2 \approx 1.489787337
\]

\section*{Задача 17.}

\noindent\textit{\textbf{Условие:}}

Функция распределения случайной величины $\xi$:
\[
F_{\xi} (x) =
\begin{cases}
0, &x \le 1 \\
1 - \frac{1}{x^2}, &x > 1
\end{cases}
\]
Найти $E\xi - ?, D\xi - ?$

\noindent\textit{\textbf{Решение:}}

Плотность распределения:
\[
p_{\xi} =
\begin{cases}
0, & x < 1 \\
\frac{2}{x^3}, &x \ge 1
\end{cases}
\]

\[
E\xi \int_{-\infty}^{\infty} x p_{\xi} (x) dx = \int_{1}^{\infty} \frac{2\cancel{x}}{x^{\cancel{3}2}} dx = \int_{1}^{\infty} \frac{2}{x^2} dx = \lim_{a \to \infty} \left(-\frac{2}{x}\right) \bigg|_{1}^a = 2
\]
\[
E\xi^2 = \int_{-\infty}^{\infty} x^2 p_{\xi}(x) dx = \int_{1}^{\infty} x^2 \frac{2}{x^3} dx = \dots = \infty
\]
\[
\Rightarrow \text{ т.к. мат. ожидание равно } \infty \Rightarrow \nexists D\xi 
\]

\end{document} 