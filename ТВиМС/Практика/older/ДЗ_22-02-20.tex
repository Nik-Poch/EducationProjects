% В этом документе преамбула

%%% Работа с русским языком
\usepackage{cmap}					% поиск в PDF
\usepackage{mathtext} 				% русские буквы в формулах
\usepackage[T2A]{fontenc}			% кодировка
\usepackage[utf8]{inputenc}			% кодировка исходного текста
\usepackage[english,russian]{babel}	% локализация и переносы
\usepackage{indentfirst}			% чтобы первый абзац в разделе отбивался красной строкой
\frenchspacing						% тонкая настройка пробелов

%%% Приведение начертания букв и знаков к русской типографской традиции
\renewcommand{\epsilon}{\ensuremath{\varepsilon}}
\renewcommand{\phi}{\ensuremath{\varphi}}			% буквы "эпсилон"
\renewcommand{\kappa}{\ensuremath{\varkappa}}		% буквы "каппа"
\renewcommand{\le}{\ensuremath{\leqslant}}			% знак меньше или равно
\renewcommand{\leq}{\ensuremath{\leqslant}}			% знак меньше или равно
\renewcommand{\ge}{\ensuremath{\geqslant}}			% знак больше или равно
\renewcommand{\geq}{\ensuremath{\geqslant}}			% знак больше или равно
\renewcommand{\emptyset}{\varnothing}				% знак пустого множества

%%% Дополнительная работа с математикой
\usepackage{amsmath,amsfonts,amssymb,amsthm,mathtools} % AMS
\usepackage{icomma} % "Умная" запятая: $0,2$ --- число, $0, 2$ --- перечисление

%% Номера формул
\mathtoolsset{showonlyrefs=true} % Показывать номера только у тех формул, на которые есть \eqref{} в тексте.

%% Свои команды

% операции, не определённые (или имеющие иные обохначения) в мат. пакетах
\DeclareMathOperator{\sgn}{\mathop{sgn}}				% ф-ия sgn
\renewcommand{\tg}{\mathop{\mathrm{tg}}\nolimits}		% обозначение тангенса

%% Перенос знаков в формулах (по Львовскому)
\newcommand*{\hm}[1]{#1\nobreak\discretionary{}
{\hbox{$\mathsurround=0pt #1$}}{}}

%%% Работа с картинками
\usepackage{graphicx}  % Для вставки рисунков
\graphicspath{{images/}{images2/}}  % папки с картинками
\setlength\fboxsep{3pt} % Отступ рамки \fbox{} от рисунка
\setlength\fboxrule{1pt} % Толщина линий рамки \fbox{}
\usepackage{wrapfig} % Обтекание рисунков текстом

%%% Работа с таблицами
\usepackage{array,tabularx,tabulary,booktabs} % Дополнительная работа с таблицами
\usepackage{longtable}  % Длинные таблицы
\usepackage{multirow} % Слияние строк в таблице

%%% Теоремы
\theoremstyle{plain} % Это стиль по умолчанию, его можно не переопределять.
\newtheorem{theorem}{Теорема}[section]
\newtheorem{lemma}{Лемма}[section]
\newtheorem{definition}[theorem]{Определение}
\newtheorem{property}{Свойство}
 
\theoremstyle{definition} % "Определение"
\newtheorem{corollary}{Следствие}[theorem]
\newtheorem{exmp}{Пример}[section]
 
\theoremstyle{remark} % "Примечание"
\newtheorem*{nonum}{Решение}
\newtheorem*{evidence}{Доказательство}
\newtheorem*{remark}{Примечание}

%%% Программирование
\usepackage{etoolbox} % логические операторы

%%% Страница
\usepackage{extsizes} % Возможность сделать 14-й шрифт
\usepackage{geometry} % Простой способ задавать поля
	\geometry{top=25mm}
	\geometry{bottom=35mm}
	\geometry{left=35mm}
	\geometry{right=20mm}

%\usepackage{fancyhdr} % Колонтитулы
% 	\pagestyle{fancy}
 	%\renewcommand{\headrulewidth}{0pt}  % Толщина линейки, отчеркивающей верхний колонтитул
% 	\lfoot{Нижний левый}
% 	\rfoot{Нижний правый}
% 	\rhead{Верхний правый}
% 	\chead{Верхний в центре}
% 	\lhead{Верхний левый}
%	\cfoot{Нижний в центре} % По умолчанию здесь номер страницы

\usepackage{setspace} % Интерлиньяж (межстрочные интервалы)
%\onehalfspacing % Интерлиньяж 1.5
%\doublespacing % Интерлиньяж 2
%\singlespacing % Интерлиньяж 1

\usepackage{lastpage} % Узнать, сколько всего страниц в документе.

\usepackage{soulutf8} % Модификаторы начертания

\usepackage{hyperref}
\usepackage[usenames,dvipsnames,svgnames,table,rgb]{xcolor}
\hypersetup{				% Гиперссылки
    unicode=true,           % русские буквы в раздела PDF
    pdftitle={Заголовок},   % Заголовок
    pdfauthor={Автор},      % Автор
    pdfsubject={Тема},      % Тема
    pdfcreator={Создатель}, % Создатель
    pdfproducer={Производитель}, % Производитель
    pdfkeywords={keyword1} {key2} {key3}, % Ключевые слова
    colorlinks=true,       	% false: ссылки в рамках; true: цветные ссылки
    linkcolor=MidnightBlue,          % внутренние ссылки
    citecolor=black,        % на библиографию
    filecolor=magenta,      % на файлы
    urlcolor=blue           % на URL
}

\usepackage{csquotes} % Еще инструменты для ссылок

%\usepackage[style=authoryear,maxcitenames=2,backend=biber,sorting=nty]{biblatex}

\usepackage{multicol} % Несколько колонок

%%% Работа с графикой
\usepackage{tikz}
\usetikzlibrary{calc}
\usepackage{tkz-euclide}
\usetikzlibrary{arrows}
\usepackage{pgfplots}
\usepackage{pgfplotstable}

%%% Настройка подписей к плавающим объектам
\usepackage{floatrow}	% размещение
\usepackage{caption}	% начертание
\captionsetup[figure]{labelfont=bf,textfont=it,font=footnotesize}	% нумерация и надпись курсивом
% для подфигур: заголовок подписи полужирный, текст заголовка обычный
% выравнивание является неровным (т.е. выровненным по левому краю)
% singlelinecheck = off означает, что настройка выравнивания используется, даже если заголовок имеет длину только одну строку.
% если singlelinecheck = on, то заголовок всегда центрируется, когда заголовок состоит только из одной строки.
\captionsetup[subfigure]{labelfont=bf,textfont=normalfont,singlelinecheck=off,justification=raggedright}

%%% Stuff для графиков и рисунков



\begin{document}
	
\textit{Почаев Никита Алексеевич, гр. 8381}

\section*{Условная вероятность (ДЗ на 22.02.20)}

\subsection*{Задача 1.}

Из колоды в 52 карты наугад выбираются 2. Определить, независимы ли события 

$P(A), P(B), P(AB), P(B|A), P(A|B)$.

\begin{enumerate}[label=\alph*)]
	\item $A - \{!T\}, B - \{!Kp\}$
	
	В данном случае порядок карт не важен $\Rightarrow \# \Omega = C_{52}^2 = \dfrac{52 \cdot 51}{2} = 1326$.
	
	Событие $A$ – $4$ варианта выбрать туз, при этом каждому из них в пару можно выбрать одну карту, не являющуюся тузом. Таких карт $52 - 4 = 48 \Rightarrow \#A = 4 \cdot 48 = 192$.
	
	Событие $B$ – 26 вариантов выбрать красную карту, при этом каждой из них в пару можно выбрать одну НЕ красную карту. Таких карт $52 - 26 = 26 \Rightarrow \#B = 26 \cdot 26 = 676$.
	
	\[ P(A) = \dfrac{192}{1326}, P(B) = \dfrac{676}{1326} \]
	
	Событие $AB$, что выпал ровно один туз и ровно одна красная карта подразделяется на два случая:
	\begin{itemize}
		\item Одна из карт – красный туз, а другая НЕ красная И НЕ туз (событие $C$)
		\item Одна из карт – НЕ красный туз, а другая – красная, но НЕ туз (событие $D$)
	\end{itemize}

	Для события $C$ есть 2 варианта выбрать красный туз и $24$ варианта выбрать черную и не туз (так как 26 черных, из которых 2 туза) $\Rightarrow \#C = 2 \cdot 24 = 48$.
	
	Для события $D$ есть 2 варианта выбрать черный туз, и 24 варианта выбрать красную, но не туза $\Rightarrow \# D = 2 \cdot 24 = 48$.
	
	\[ P(AB) = P(C) + P(D) = \dfrac{48 + 48}{1326} = \dfrac{16}{221} \]
	
	Проверка независимости событий.
	
	\[ P(A) \cdot P(B) \approx 0.07382 \ne P(AB) \Rightarrow \text{ события } A \text{ и } B \text{ \textbf{зависимы}} \]
	\[ P(A|B) = \dfrac{P(AB)}{P(B)} \approx \dfrac{0.0724}{0.5098} \approx 0.142 \cdot \left( \dfrac{96}{676} \right) \]
	\[ P(B|A) = \dfrac{P(AB)}{P(A)} \approx \dfrac{0.0724}{0.14479} \approx 0.5 \cdot \left( \dfrac{96}{192} \right) \]
	
	\item $A - \{!\text{Т пик }\}, B - \{ !\text{Д черви} \}$
	\[ \# \Omega = C_{52}^2 = \dfrac{52 \cdot 51}{2} = 1326 \]
	
	Событие $A$ – туз пик + одна из оставшихся 51 карт: $\# A = 51, P(A) = \dfrac{51}{1326} = \dfrac{1}{26}$.
	
	Событие $B$ – дама черви + одна из оставшихся 51 карт: $\# B = 51, P(B) = \dfrac{51}{1326} = \dfrac{1}{26}$
	
	Событие $AB:$ одна карта – туз пик, а вторая – дама черви (одна карта не может быть одновременно и тем, и тем).
	\[ P(AB) = \dfrac{1}{1326} = \dfrac{1}{\frac{51 \cdot 52}{2}} \ne P(A) \cdot P(B) \]
	$\Rightarrow$ события \textbf{зависимы}.
	
	\[ P(A|B) = P(B|A) = \dfrac{P(AB)}{P(A)} = \dfrac{26}{51 \cdot 26} = \dfrac{1}{51} = 0.0196 \]
	
	
	\item $A - \{\text{обе красные}\}, B - \{!\text{Т пик}\}$
	\[ \# \Omega = C_{52}^2 = \dfrac{52 \cdot 51}{2} = 1326 \]
	
	Событие $A$ – взять две красные карты, не учитывая порядок, т.к. была выбрана модель без учета порядка.
	\[ \# A = C_{26}^2 = 325, P(A) = \dfrac{325}{1326} = \dfrac{25}{102} \]
	
	Событие $B$ - один туз пик, а вторую карту можно выбрать 51 способам.
	\[ \# B = 51, P(B) = \dfrac{51}{1326} = \dfrac{1}{26} \]
	
	Событие $AB$ – одна из карт – туз пик, но тогда невозможно, чтобы было 2 красные карты $\Rightarrow P(AB) = 0$.
	
	$\Rightarrow$ события \textbf{зависимы} и $P(A|B) = P(B|A) = 0$.
	
	\item $A - \{\text{обе красные}\}, B - \{\text{одной масти}\}$
	\[ \# \Omega = C_{52}^2 = \dfrac{52 \cdot 51}{2} = 1326 \]
	
	Событие $A$ – взять две красные карты, не учитывая порядок, т.к. была выбрана модель без учета порядка.
	\[ \# A = C_{26}^2 = 325, P(A) = \dfrac{325}{1326} = \dfrac{25}{102} \]
	
	Событие $B$ – взять карты одной масти, не учитывая порядок. Всего мастей 4, и в каждой 13 карт.
	\[ \# B = 4 \cdot C_{13}^2 = 78 \cdot 4, P(B) = \dfrac{78 \cdot 4}{1326} = \dfrac{4}{17} \]
	
	Событие $AB$ - берутся две красные карты одной масти, а чтобы выполнить $A$ - две карты красной масти.
	\[ P(AB) = \dfrac{2 \cdot C_{13}^2}{1326} = \dfrac{2}{17} \ne P(A) \cdot P(B) \]
	$\Rightarrow$ события \textbf{зависимы}.
	\[ P(A|B) = \dfrac{P(AB)}{P(B)} = 0.5 \]
	\[ P(B|A) = \dfrac{P(AB)}{P(A)} = 0.48 \]
\end{enumerate}

\subsection*{Задача 2.}

Монета бросается трижды. Определить независимы ли события: 

$A - \{1 - 0\}, B - \{3 - p\}, C - \{o > p\}, D - \{! - o\}, E - \{\text{хоть! о}\}, F - \{\ge 2 o\}$

Если зависимы, посчитать зависимость одного от другого. Цифра $\to$ упорядоченный набор. Но можно считать иначе, если строим другую модель.

\textit{Решение:}

Порядок подбрасывания монет по условию важен, а повторения возможны $\Rightarrow$ упорядоченный набор с повторениям.
\[ \# \Omega = 2^3 = 8 \]

\begin{enumerate}[label=\alph*)]
	\item Проверим независимость $A$ и $B$.
	
	Событие $A$ - 1-ый раз выпал орёл, 2-ой и 3-ий - всё, что угодно $\Rightarrow \# A = 1 \cdot 2 \cdot 2 = 4, P(A) = \dfrac{1}{2}$
	
	Событие $B$ - 1-ый и 2-ой раз выпало всё, что угодно, а в 3-ий - решка $\Rightarrow \# B = 2 \cdot \cdot 2 \cdot 1 = 4, P(B) = \dfrac{1}{2}$
	
	Событие $AB$ - 1-ый раз выпал орёл, 2-ой раз всё, что угодно, а 3-ий - решка $\Rightarrow \# AB = 1 \cdot 2 \cdot 2 = 2, P(AB) = \dfrac{1}{4}$
	\[ P(AB) = P(A) \cdot P(B) \]
	$\Rightarrow$ события \textbf{независимы}.
	
	\item Проверим независимость $A$ и $C$.
	
	Событие $A$ описано в предыдущем пункте.
	
	Событие $C$ - за три броска выпало больше орлов, чем решек. Два орла за 3 броска может выпасть $C_3^2 = 3$ способами, также подходящим является выпадение 3-х орлов.
	\[ \# C = 3 + 1 = 4, P(C) = \dfrac{1}{2} \]
	
	Событие $AC$ - первым выпал орёл и орлов больше, чем решек. Т.к. первым выпал орёл, то бросок, в который выпадет 2-ой, можно выбрать двумя способами + исход, когда выпадает два орла.
	\[ \# AC = 2 + 1 = 3, P(AC) = \dfrac{3}{8} \ne P(A) \cdot P(C) \]
	
	$\Rightarrow$ события \textbf{зависимы}.
	\[ P(A|C) = P(C|A) = \dfrac{P(AC)}{P(A)} = \dfrac{3 \cdot 2}{8} = 0.75 \]
	
	\item Проверим независимость $B$ и $C$.
	
	Событие $B$ описано в пункте a), событие $C$ - в пункте b).
	\[ P(B) = \dfrac{1}{2}, P(C) = \dfrac{1}{2} \]
	
	Событие $BC$ - третьим броском выпала решка и орлов за все броски больш, чем решек. Первыми двумя бросками должны выпасть орлы $\Rightarrow$
	\[ P(BC) = \dfrac{1}{8} \ne P(B) \cdot P(C) = \dfrac{1}{4} \]
	
	$\Rightarrow$ события \textbf{зависимы}.
	\[ P(B|C) = P(C|B) = \dfrac{P(BC)}{P(B)} = \dfrac{2}{8} = \dfrac{1}{4} \]
	
	\item Проверим независимость $(A \cup B)$ и $C$.
	
	События $A, B, C$ и $AB$ рассмотрены в предыдущих пунктах.
	
	\[ P(A \cup B) = P(A) + P(B) - P(AB) = \dfrac{1}{2} + \dfrac{1}{2} - \dfrac{1}{4} = \dfrac{3}{4} \]
	
	Событие $(A \cup B)$ и $C$ - первый орёл, либо третья решка, но обязательно орлов больше, чем решек. Всего таких вариантов 3: ОРО, ООО, ООР.
	\[ P\left( (A \cup B) \cap C \right) = \dfrac{3}{8} = P(A \cup B) \cdot P(C) \]
	
	$\Rightarrow$ события \textbf{независимы}.
	
	\item Проверим независимость $(A \cup B)$ и $(B \cup C)$.
	
	События $A, B, C$ рассмотрены в предыдущих пунктах.
	
	\[ P(A \cup B) = \dfrac{3}{4} \]
	\[ P(B \cup C) = P(B) + P(C) - P(BC) = \dfrac{1}{2} + \dfrac{1}{2} - \dfrac{1}{8} = \dfrac{7}{8} \]
	
	Событие $(A \cup B)$ и $(B \cup C)$ - (первый орёл ИЛИ третья решка) И (третья решка ИЛИ орлов больше, чем решек).
	\begin{itemize}
		\item По св-ву ассоциативности операции объединения множеств ($(A \cup B) \cup C = A \cup (B \cup C)$), если выполнено условие $B$, то событие обязательно случиться.
		
		$\Rightarrow$ берём все случаи, когда третья решка, т.е. $\# B = 4$. 
		
		\item Рассматриваем случаи, когда третьим броском выпадает не решка. Первый должен быть орёл $\Rightarrow$ ООО или ОРО.
	\end{itemize}

	\[ P \left( (A \cup B) \cap (B \cup C) \right) = \# B + 2 = 4 + 2 = 6 \]
	\[ P \left( (A \cup B) \cap (B \cup C) \right) = \dfrac{6}{8} = \dfrac{3}{4} \ne P(A \cup B) \cdot P(B \cup C) \]
	
	$\Rightarrow$ события \textbf{зависимы}.
	\[ P \left( (A \cup B) | (B \cup C) \right) = \dfrac{6}{7} \]
	\[ P \left( (B \cup C) | (A \cup B) \right) = 1 \]
	
	\item Проверим независимость $(A \cup B)$ и $D$.
	
	События $A$ и $B$ рассмотрены в предыдущих пунктах.
	\[ P(A \cup B) = \dfrac{3}{4} \]
	
	Событие $D$ - выпал ровно один орёл. Тогда таких исходов: $C_3^1 = 3$.
	\[ P(D) = \dfrac{3}{8} \]
	
	Событие $(A \cup B)$ и $D$ - выпал ровно один орёл, при этом либо 1-ый орёл, либо 3-я решка. Удовлетворяющие варианты: ОРР, РОР.
	\[ P \left( (A \cup B) \cap D \right) = \dfrac{2}{8} = \dfrac{1}{4} \ne \dfrac{3}{4} \cdot \dfrac{3}{8} \]
	
	$\Rightarrow$ события \textbf{зависимы}.
	\[ P \left( (A \cup B) | D = \dfrac{2}{3} \right) \]
	\[ P \left( D | (A \cup B) \right) = \dfrac{1}{3} \]
	
	\item Проверим независимость $D$ и $(E \cap F)$
	
	Событие $D$ – выпал ровно один орел.
	\[ P(D) = \dfrac{C_3^1}{8} = \dfrac{3}{8} \]
	
	Событие $E$ – выпал хоть один орел. От обратного – не выпадет ни одного орла один раз:
	\[ P(E) = 1 - P(\bar E) = 1 - \dfrac{1}{8} = \dfrac{7}{8} \]
	
	Событие $F$ – выпало два орла или более. Подходящие исходы: ООР, ОРО, РОО, ООО. $\# F = 4$.
	\[ P(F) = \dfrac{4}{8} = \dfrac{1}{2} \]
	
	Событие $E \cap F$ - выпал хоть один орел и выпало два орла и более. Если выпало два орла или более, то хоть один орел точно выпал $\Rightarrow$
	\[ P(EF) = P(F) = \dfrac{1}{2} \]
	
	Событие $\left( D \cap (E \cap F) \right) \eq (D \cap E \cap F)$. Если выпало два орла или более, то ровно один орел не мог выпасть, поэтому таких случаев нет, т.е.
	\[ P \left( D \cap E \cap F \right) = 0 \ne P(D) \cdot P(EF) \]
	
	$\Rightarrow$ события \textbf{независимы}.
	
	Из того, что одно случается вытекает, что другое точно не случится.
	\[ P(D|(E \cap F)) = P((E \cap F) | D) = 0 \]
	
	\item Проверим независимость $D$ и $E$.
	
	События $D$ и $E$ рассмотрены в предыдущих пунктах.
	
	Событие ($D$ и $F$) - выпал ровно один орел и выпал хоть один орел. Если выпал ровно один орел, то хоть один орел точно выпал $\Rightarrow$
	\[ P(D \cap E) = P(D) = \dfrac{3}{8} \ne P(D) \cdot P(E) \]
	
	$\Rightarrow$ события \textbf{зависимы}.
	\[ P(E|D) = 1, P(D|E) = \dfrac{3}{7} \]
\end{enumerate}

\subsection*{Задача 3.} 

В соревнованиях участвуют 8 - команд: 4 из премьер лиги и 4 из футбольной национальной лиги. Образуются 4 пары, необходимо определить вероятность, что каждой команде из ПЛ будет поставлена команда из ФНЛ.

\textit{Решение:}

Рассмотрим последовательный выбор всех пар ПЛ и ФНЛ.

При выборе первой пары множество всех исходов по выбору 2-х команд равно: $\# \Omega_1 = C_8^2 = 28$. Пусть событие $A_1$ - сформирована требуемая пара "ПЛ + ФНЛ"\, тогда $\exists$ 4 способа выбрать 1-ю команду и аналогично 4 выбрать 2-ю $\Rightarrow \# A_1 = 4 \cdot 4 = 16$.
\[ P(A_1) = \dfrac{16}{28} = \dfrac{4}{7} \]

Событие $(A_2|A_1)$ - сформирована 2-ая требуемая пара, при условии, что 1-ая уже создана. Т.к. в ПЛ и ФНЛ -1 команда $\Rightarrow \# \Omega_2 = C_6^2 = 15$. По тем же соображениям: $\# (A_2|A_1) = 3 \cdot 3 = 9$. 
\[ P((A_2|A_1)) = \dfrac{9}{15} \]

Вероятность \textbf{совместного появления двух зависимых событий} равна произведению вероятности одного из них на условную вероятность второго, вычисленную при условии, что первое событие произошло, т.е. $P(AB) = P(B) \cdot P(A|B) = P(A) \cdot P(B|A)$.

Таким образом, $P(A_1A_2) = P(A_2|A_1) \cdot P(A_1) = \dfrac{3}{5} \cdot \dfrac{4}{7} = \dfrac{12}{35}$

Событие $(A_3|A_1A_2)$ - сформировалась 3-я пара, при условии создания первых 2-х. Аналогично предыдущему шагу, $\# \Omega_3 = C_4^2 = 6, \# A_3 = 2 \cdot 2 = 4$.
\[ P((A_3|A_1A_2)) = \dfrac{2}{3} \]

\[ P(A_1A_2A_3) = P((A_3|A_1A_2)) \cdot P(A_1A_2) = \dfrac{2}{3} \cdot \dfrac{12}{35} = \dfrac{8}{35} \]

Событие $(A_4|A_1A_2A_3)$ - сформировалась 4-ая требуемая пара. Очевидно, что $\# \Omega_4 = 1, \# (A_4|A_1A_2A_3) = 1 \Rightarrow P((A_4|A_1A_2A_3)) = 1$.

\[ P(A_1A_2A_3A_4) = P((A_4|A_1A_2A_3)) \cdot P(A_1A_2A_3) = \dfrac{8}{35} \]

\subsection*{Задача 4.}

Товар стоит 50 рублей. В очереди стоят только люди, у кого 50 и 100 рублей. $n$ - 100 р., $n$ - 50 р. Очередь выстроена в случайном порядке. Найти вероятность, что очередь, обслужена по порядку, если в начальный момент в кассе нет денег.

\textit{Решение:}

Пусть $X$ - купюра в 50 р., а $Y$ - купюра в 100 р.

В начале очереди может стоять только покупатель с $X$, т.к. иначе сразу не на что давать сдачу, т.е. не подходит вариант:

\begin{table}[h!]
	\begin{tabular}{|l|l|l|l|l|}
		$Y$ & $x/y$ & $x/y$ & $\dots$ & $x/y$ \\ \hline
	\end{tabular}
\end{table}

Рассмотрим варианты расположения 1-ой встреченной $Y$ в очереди. Пусть $A_1$ - событие, при котором для неё оказалась сдача. По условию кол-во $Y = X = n \Rightarrow$ после 1-ой стоит ещё $n-1$ купюра $Y$ и значит позиция 1-ой: от 2 до $n+1$:

\begin{table}[H]
	\begin{tabular}{cccccl}
		\multicolumn{1}{|c|}{$X$}                & \multicolumn{1}{c|}{$Y$}     & \multicolumn{1}{c|}{$x/y$}   & \multicolumn{1}{c|}{$x/y$}   & \multicolumn{1}{c|}{$\dots$} & \multicolumn{1}{l|}{$x/y$}   \\ \hline
		&                              &                              &                              &                              &                              \\
		\multicolumn{1}{|c|}{$X$}                & \multicolumn{1}{c|}{$X$}     & \multicolumn{1}{c|}{$Y$}     & \multicolumn{1}{c|}{$x/y$}   & \multicolumn{1}{c|}{$\dots$} & \multicolumn{1}{l|}{$x/y$}   \\ \hline
		&                              &                              &                              &                              &                              \\
		\multicolumn{1}{|c|}{$\dots$}            & \multicolumn{1}{c|}{$\dots$} & \multicolumn{1}{c|}{$\dots$} & \multicolumn{1}{c|}{$\dots$} & \multicolumn{1}{c|}{$\dots$} & \multicolumn{1}{l|}{$\dots$} \\ \hline
		&                              &                              &                              &                              &                              \\
		\multicolumn{1}{|c|}{$(XX \dots X)_{n}$} & \multicolumn{1}{c|}{$Y$}     & \multicolumn{1}{c|}{$x/y$}   & \multicolumn{1}{c|}{$x/y$}   & \multicolumn{1}{c|}{$\dots$} & \multicolumn{1}{l|}{$x/y$}   \\ \hline
	\end{tabular}
\end{table}

Всего вариантов разместить 1-ю $Y$:
\[ \# \Omega = n + 1 \]
\[ P(A_1) = \dfrac{n}{n+1} \]

Событие $(A_2|A_1)$ - для 2-ой $Y$ оказалась сдача, при условии, что для 1-ой она нашлась.

Воспользуемся тем фактом, что дальнейшего влияния на последовательность пара $X$ и $Y$ не оказывает, поэтому уберём их из построенной модели. В результате остаётся $X: (n-1)$ и $Y: (n-1)$.

По аналогичным рассуждениям позиция 2-ого $Y$ может быть от $2$ до $(n-1)+1$. Тогда $\# \Omega_2 = n$.
\[ P((A_2|A_1)) = \dfrac{n-1}{n} \]

\[ P(A_1A_2) = P(A_2|A_1) \cdot P(A_1) = \dfrac{n}{n+1} \cdot \dfrac{n-1}{n} \]

Далее можем получить:

\[ P(A_3|A_1A_2) = \dfrac{n-2}{n-1} \]

\[ P(A_1A_2A_3) = \dfrac{n}{n+1} \cdot \dfrac{n-1}{n} \cdot \dfrac{n-2}{n-1} \]

Для $A_n$ формула примет вид:

\[ P(A_1A_2 \dots A_n) = \prod_{i = 1}^{n} \dfrac{i}{i + 1} \]

\end{document} % конец документа